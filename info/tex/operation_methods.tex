\section{Методика работы}

%%---------------------------------------------------------------------------------------------------
  \pointbold{Требования к изделию}
  
    \subpoint Работы по данной инструкции допускается проводить с платой \DocProductShortTitle, соответствующей сборочному чертежу \DocProductSignature~ СБ. % ??? и принятой ОТК.
    
    \begin{footnotesize}    
      Примечание --- Все наименования цепей и элементов указаны согласно схеме электрической принципиальной \-\DocProductSignature~ Э3.      
    \end{footnotesize}  
  
%%---------------------------------------------------------------------------------------------------
  \pointbold {Технологическая вибрация}
    \subpoint \DocProductShortTitle~ должен быть подвергнут воздействию технологической вибрации с помощью вибростенда
    в соответствии с действующей конструкторской документацией (паспорт технологический ИЯДС.466226.002 Д).
    
    \subpoint После воздействия технологической вибрации следует визуально удостовериться в отсутствии дефектов монтажа.
    
%%---------------------------------------------------------------------------------------------------  
  \pointbold {Проверка целостности цепей питания и вторичных источников питания}
  
  \subpoint Перед установкой перемычек, соединяющих выходы вторичных источников питания \DocProductShortTitle~ с нагрузкой,
  с помощью мультиметра проверьте сопротивление цепей питания \DocProductShortTitle~ относительно общей точки схемы (цепи GND).  
  Измеренные сопротивления должны находиться в пределах, указанных в таблице \ref{tab:test_r}.
  \label {test_R} %Начало описания проверки сопротивлений цепей питания
  
  %%Вставка таблицы \ref{tab:test_r}
    %% Таблица со значениями сопротивлений
  \newcommand{\ltheadtestR}{}
  \renewcommand{\ltheadtestR}{
    \hline
    \multicolumn{1}{m{3.5cm}|}{\centering Имя\-/описание цепи}&
    \multicolumn{1}{m{3.0cm}|}{\centering Кон\-троль\-ная точка №1 (вывод компонента)}&
    \multicolumn{1}{m{3.0cm}|}{\centering Кон\-троль\-ная точка №2 (вывод компонента)}&    
    \multicolumn{1}{m{3.0cm}|}{\centering Минимальное значение сопротивления, Ом}&
    \multicolumn{1}{m{3.0cm} }{\centering Максимальное значение сопротивления, Ом}\\
    \hline}  
  \begin{longtable}{m{3.5cm}|m{3.0cm}|m{3.0cm}|m{3.0cm}|m{3.0cm}}
    \caption{Сопротивления цепей питания}
    \label{tab:test_r}\\
    % первая шапка     
    \ltheadtestR
    \endfirsthead   
    % последующие шапки 
    \caption*{\it{Продолжение таблицы} \thetable}\\
    \ltheadtestR
    \endhead
    % концевики
    %\multicolumn{3}{c}{\hidehline}
    \endfoot
    \endlastfoot
    %============ содержимое таблицы==============================
    %%Строка
    \multicolumn{1}{m{3.5cm}|}{\centering Выход ИП VCC3.3}&
    \multicolumn{1}{m{3.0cm}|}{\centering K109}&
    \multicolumn{1}{m{3.0cm}|}{\centering вывод №1 колодки XP1}&    
    \multicolumn{1}{m{3.0cm}|}{\centering 15800}&
    \multicolumn{1}{m{3.0cm} }{\centering 23600}\\\hline
     %%Строка
    \multicolumn{1}{m{3.5cm}|}{\centering VCC3.3}&
    \multicolumn{1}{m{3.0cm}|}{\centering K78}&
    \multicolumn{1}{m{3.0cm}|}{\centering вывод №1 колодки XP1}&    
    \multicolumn{1}{m{3.0cm}|}{\centering 700}&
    \multicolumn{1}{m{3.0cm} }{\centering 1000}\\\hline
    %%Строка
    \multicolumn{1}{m{3.5cm}|}{\centering Выход ИП VCC2.5}&
    \multicolumn{1}{m{3.0cm}|}{\centering K123}&
    \multicolumn{1}{m{3.0cm}|}{\centering вывод №1 колодки XP1}&    
    \multicolumn{1}{m{3.0cm}|}{\centering 3400}&
    \multicolumn{1}{m{3.0cm} }{\centering 5200}\\\hline
    %%Строка
    \multicolumn{1}{m{3.5cm}|}{\centering VCC2.5}&
    \multicolumn{1}{m{3.0cm}|}{\centering K122}&
    \multicolumn{1}{m{3.0cm}|}{\centering вывод №1 колодки XP1}&    
    \multicolumn{1}{m{3.0cm}|}{\centering 1300}&
    \multicolumn{1}{m{3.0cm} }{\centering 2000}\\\hline    
    %%Строка
    \multicolumn{1}{m{3.5cm}|}{\centering Выход ИП VCC1.0}&
    \multicolumn{1}{m{3.0cm}|}{\centering K128}&
    \multicolumn{1}{m{3.0cm}|}{\centering вывод №1 колодки XP1}&    
    \multicolumn{1}{m{3.0cm}|}{\centering 3500}&
    \multicolumn{1}{m{3.0cm} }{\centering 5300}\\\hline
    %%Строка
    \multicolumn{1}{m{3.5cm}|}{\centering VCC1.0}&
    \multicolumn{1}{m{3.0cm}|}{\centering K127}&
    \multicolumn{1}{m{3.0cm}|}{\centering вывод №1 колодки XP1}&    
    \multicolumn{1}{m{3.0cm}|}{\centering 9,8}&
    \multicolumn{1}{m{3.0cm} }{\centering 14,8}\\\hline
    %%Строка
    \multicolumn{1}{m{3.5cm}|}{\centering Выход ИП DDR3\_VDD}&
    \multicolumn{1}{m{3.0cm}|}{\centering K116}&
    \multicolumn{1}{m{3.0cm}|}{\centering вывод №1 колодки XP1}&    
    \multicolumn{1}{m{3.0cm}|}{\centering 4400}&
    \multicolumn{1}{m{3.0cm} }{\centering 4900}\\\hline
    %%Строка
    \multicolumn{1}{m{3.5cm}|}{\centering DDR3\_VDD}&
    \multicolumn{1}{m{3.0cm}|}{\centering K117}&
    \multicolumn{1}{m{3.0cm}|}{\centering вывод №1 колодки XP1}&    
    \multicolumn{1}{m{3.0cm}|}{\centering 1400}&
    \multicolumn{1}{m{3.0cm} }{\centering 1700}\\\hline
    %%Строка
    \multicolumn{1}{m{3.5cm}|}{\centering Выход ИП DDR3\_VTT}&
    \multicolumn{1}{m{3.0cm}|}{\centering K114}&
    \multicolumn{1}{m{3.0cm}|}{\centering вывод №1 колодки XP1}&    
    \multicolumn{1}{m{3.0cm}|}{\centering 1 700 000}&
    \multicolumn{1}{m{3.0cm} }{\centering 2 600 000}\\\hline
    %%Строка
    \multicolumn{1}{m{3.5cm}|}{\centering DDR3\_VTT}&
    \multicolumn{1}{m{3.0cm}|}{\centering K115}&
    \multicolumn{1}{m{3.0cm}|}{\centering вывод №1 колодки XP1}&    
    \multicolumn{1}{m{3.0cm}|}{\centering 1700}&
    \multicolumn{1}{m{3.0cm} }{\centering 2600}\\\hline
    %%Строка
    \multicolumn{1}{m{3.5cm}|}{\centering Выход ИП DDR3\_VREF}&
    \multicolumn{1}{m{3.0cm}|}{\centering K112}&
    \multicolumn{1}{m{3.0cm}|}{\centering вывод №1 колодки XP1}&    
    \multicolumn{1}{m{3.0cm}|}{\centering 3 900 000}&
    \multicolumn{1}{m{3.0cm} }{\centering 4 300 000}\\\hline
    %%Строка
    \multicolumn{1}{m{3.5cm}|}{\centering DDR3\_VREF}&
    \multicolumn{1}{m{3.0cm}|}{\centering K113}&
    \multicolumn{1}{m{3.0cm}|}{\centering вывод №1 колодки XP1}&    
    \multicolumn{1}{m{3.0cm}|}{\centering 156 000}&
    \multicolumn{1}{m{3.0cm} }{\centering 290 000}\\\hline
    %%Строка
    %%Исключаем строку из-за отсутствия достоверных измерений
    %\multicolumn{1}{m{3.5cm}|}{\centering +1.8V\_CPLD, после фильтра}&
    %\multicolumn{1}{m{3.0cm}|}{\centering K83}&
    %\multicolumn{1}{m{3.0cm}|}{\centering вывод №1 колодки XP1}&    
    %multicolumn{1}{m{3.0cm}|}{\centering ???}&
    %multicolumn{1}{m{3.0cm} }{\centering ???}\\\hline  
  \end{longtable}
  %%---------------------------------------------------------------------------------------------------  
  
  \subpoint Если обнаружено несоответствие, то следует определить причину неисправности и устранить её, затем повторить действия, начиная с пункта~ \ref{test_R}.  
  
  \subpoint Включите источник питания А1 из состава \DocStendShortTitle (далее по тексту --- источник питания A1).
  \subpoint Убедитесь, что выходные параметры канала №1 источника питания настроены следующим образом:
  \begin{itemize}
    \item выходное напряжение: (12,0~$\pm$~0,1)~В;
    \item ограничение по току: не менее 1,0~А.
  \end{itemize}
  \subpoint Если выходные параметры канала №1 источника питания отличаются от приведённых, то установите их согласно пункту 3.2.4 руководства по эксплуатации \DocStendSignature~ РЭ.
    
  \subpoint Отключите источник питания A1.
  \subpoint Припаяйте два провода типа МГТФ 0,35 длиной (0,5~$\pm$~0,2) м 
  в соответствии со схемой~ \ref{ris:vim_power_connection} и рисунком~ \ref{ris:vim_vcc12_adapter_solder_foto} приложения~ \ref{appendix:power_adapter} 
  к следующим точкам платы \DocProductShortTitle:
  \begin{itemize}
   \item контакт предохранителя FU2 (цепь VS1);
   \item анод защитного диода VD23 (цепь GND).
  \end{itemize}
  \subpoint Свободные концы проводов залудить на длину не менее 10 мм и присоединить к клеммам канала №1 источника питания
  в соответствии со схемой~ \ref{ris:vim_power_connection} приложения~ \ref{appendix:power_adapter}.
  
  %Технологическая перемычка нужна долько для программирования ПЛИС вне стенда. Исключаем (?временно) этот пункт
  %\subpoint Припаяйте технологическую перемычку между контактом резистора R434 (цепь +3.3V\_AUX) и контактом резистора R439 (цепь VCC3.3) в соответствии со схемой и рисунком приложения~ \ref{appendix:1}.
  
  \subpoint Проверьте правильность функционирования вторичного источника питания 3,3~В (цепь VCC3.3) \DocProductShortTitle~ без нагрузки.
  \label {test_vcc33} %Начало описания проверки ИП vcc33    
    \subsubpoint Включите источник питания A1.    
    \subsubpoint С помощью мультиметра измерьте значение выходного напряжения вторичного источника питания 3,3~В (цепь VCC3.3) \DocProductShortTitle~
		  относительно общей точки схемы (цепь GND). 
		  Измерение проводить между контрольными точками, указанными в таблице \ref{tab:test_v}.    
    \subsubpoint Отключите источник питания A1.    
    \subsubpoint Если измеренное напряжение находится в пределах, указанных в таблице \ref{tab:test_v}, то считается, что вторичный источник питания правильно функционирует без нагрузки.    
    \subsubpoint Если вторичный источник питания правильно функционирует без нагрузки, установите перемычку между точками, 
    указанными в таблице \ref{tab:power_jumper_points}, в соответствии с \DocProductSignature~СБ.
    
    \begin{footnotesize}    
      Примечание --- Дополнительно перемычки, установленные по цепям питания, изображены на рисунке ~\ref{ris:vim_power_jumper_foto}~ приложения ~\ref{appendix:power_jumpers}.
    \end{footnotesize}  
        
    \subsubpoint Если обнаружено несоответствие, то следует определить причину неисправности и устранить её, затем повторить действия, начиная с пункта~ \ref{test_vcc33}.
  
  \begin{sloppypar}
  \subpoint Проверьте правильность функционирования вторичных источников питания \DocProductShortTitle~ 
	    2,5~В (цепь VCC2.5), 1~В (цепь VCC1.0), 1,5~В (цепь DDR3\_VDD), 0,75~В (цепь DDR3\_VTT), 0,75~В (цепь DDR3\_VREF) без нагрузки.
  \end{sloppypar}  
  \label {test_vcc25_vcc10_ddrpower} %Начало описания проверки ИП
  
    \subsubpoint Включите источник питания A1.
    \begin{sloppypar}
    \subsubpoint С помощью мультиметра измерьте значение выходного напряжения вторичых источников питания \DocProductShortTitle~ 
		  2,5 В (цепь VCC2.5), 1 В (цепь VCC1.0), 1,5 В (цепь DDR3\_VDD), 0,75 В (цепь DDR3\_VTT), 0,75 В (цепь DDR3\_VREF) относительно общей точки схемы (цепь GND). 
		  Измерение проводить между контрольными точками, указанными в таблице \ref{tab:test_v}.        
    \end{sloppypar}
    \subsubpoint Отключите источник питания A1.
    \subsubpoint Если измеренные напряжения находятся в пределах, указанных в таблице \ref{tab:test_v}, то считается, что вторичные источники питания правильно функционируют без нагрузки.
    \subsubpoint Если вторичные источники питания правильно функционируют без нагрузки, установите перемычки между точками, указанными в таблице \ref{tab:power_jumper_points}, 
    в соответствии с \DocProductSignature~СБ.
    
    \begin{footnotesize}    
      Примечание --- Дополнительно перемычки, установленные по цепям питания, изображены на рисунке ~\ref{ris:vim_power_jumper_foto}~ приложения ~\ref{appendix:power_jumpers}.
    \end{footnotesize}  
		  
    \subsubpoint Если обнаружено несоответствие, то следует определить причину неисправности и устранить её, затем повторить действия, начиная с пункта~\ref{test_vcc25_vcc10_ddrpower}.
  
  %Вставка таблицы ref{tab:test_v}
  %% Таблица со значениями напряжений
\newcommand{\ltheadtestV}{}
\renewcommand{\ltheadtestV}{
\hline
\multicolumn{1}{m{3.0cm}|}{\centering Имя\-/описание цепи}&
\multicolumn{1}{m{2.5cm}|}{\centering Кон\-троль\-ная точка №1 (вывод компонента)}&
\multicolumn{1}{m{2.5cm}|}{\centering Кон\-троль\-ная точка №2 (вывод компонента)}&    
\multicolumn{1}{m{2.3cm}|}{\centering Ми\-ни\-маль\-ное напряжение, В}&
\multicolumn{1}{m{2.3cm}|}{\centering Мак\-си\-мальн\-ое напряжение, В}&
\multicolumn{1}{m{2.3cm}}{\centering Диапазон значений напряжения в \%}\\
\hline}

\begin{longtable}{m{3.0cm}|m{2.5cm}|m{2.5cm}|m{2.3cm}|m{2.3cm}|m{2.3cm}}
\caption{Допустимые значения выходного напряжения вторичных источников питания}
\label{tab:test_v}\\
% первая шапка     
\ltheadtestV
\endfirsthead   
% последующие шапки 
\caption*{\it{Продолжение таблицы} \thetable}\\
\ltheadtestV
\endhead
% концевики
%\multicolumn{3}{c}{\hidehline}
\endfoot
\endlastfoot
%============ содержимое таблицы==============================
%%Строка
\multicolumn{1}{m{3.0cm}|}{\centering VCC3.3}&
\multicolumn{1}{m{2.5cm}|}{\centering K109}&
\multicolumn{1}{m{2.5cm}|}{\centering вывод №1 колодки XP1}&    
\multicolumn{1}{m{2.3cm}|}{\centering 3,201}&
\multicolumn{1}{m{2.3cm}|}{\centering 3,399}&
\multicolumn{1}{m{2.3cm}}{\centering $\pm$~3}\\\hline
%%Строка
\multicolumn{1}{m{3.0cm}|}{\centering VCC2.5}&
\multicolumn{1}{m{2.5cm}|}{\centering K123}&
\multicolumn{1}{m{2.5cm}|}{\centering вывод №1 колодки XP1}&    
\multicolumn{1}{m{2.3cm}|}{\centering 2,425}&
\multicolumn{1}{m{2.3cm}|}{\centering 2,575}&
\multicolumn{1}{m{2.3cm}}{\centering $\pm$~3}\\\hline
%%Строка
\multicolumn{1}{m{3.0cm}|}{\centering VCC1.0}&
\multicolumn{1}{m{2.5cm}|}{\centering K128}&
\multicolumn{1}{m{2.5cm}|}{\centering вывод №1 колодки XP1}&    
\multicolumn{1}{m{2.3cm}|}{\centering 0,97}&
\multicolumn{1}{m{2.3cm}|}{\centering 1,03}&
\multicolumn{1}{m{2.3cm}}{\centering $\pm$~3}\\\hline
%%Строка
\multicolumn{1}{m{3.0cm}|}{\centering DDR3\_VDD}&
\multicolumn{1}{m{2.5cm}|}{\centering K116}&
\multicolumn{1}{m{2.5cm}|}{\centering вывод №1 колодки XP1}&    
\multicolumn{1}{m{2.3cm}|}{\centering 1,455}&
\multicolumn{1}{m{2.3cm}|}{\centering 1,545}&
\multicolumn{1}{m{2.3cm}}{\centering $\pm$~3}\\\hline
%%Строка
\multicolumn{1}{m{3.0cm}|}{\centering DDR3\_VTT}&
\multicolumn{1}{m{2.5cm}|}{\centering K114}&
\multicolumn{1}{m{2.5cm}|}{\centering вывод №1 колодки XP1}&    
\multicolumn{1}{m{2.3cm}|}{\centering 0,728}&
\multicolumn{1}{m{2.3cm}|}{\centering 0,773}&
\multicolumn{1}{m{2.3cm}}{\centering $\pm$~3}\\\hline
%%Строка
\multicolumn{1}{m{3.0cm}|}{\centering DDR3\_VREF}&
\multicolumn{1}{m{2.5cm}|}{\centering K112}&
\multicolumn{1}{m{2.5cm}|}{\centering вывод №1 колодки XP1}&    
\multicolumn{1}{m{2.3cm}|}{\centering 0,728}&
\multicolumn{1}{m{2.3cm}|}{\centering 0,773}&
\multicolumn{1}{m{2.3cm}}{\centering $\pm$~3}\\\hline
%%Строка
\multicolumn{1}{m{3.0cm}|}{\centering +1.8V\_CPLD}&
\multicolumn{1}{m{2.5cm}|}{\centering K83}&
\multicolumn{1}{m{2.5cm}|}{\centering вывод №1 колодки XP1}&    
\multicolumn{1}{m{2.3cm}|}{\centering 1,725}&
\multicolumn{1}{m{2.3cm}|}{\centering 1,860}&
\multicolumn{1}{m{2.3cm}}{\centering \(+3,3\) \\ \(-4,2\)}\\\hline
\end{longtable}  
    
  %Вставка таблицы ref{tab:power_jumper_points}
  %% Таблица с контрольными точками для установки перемычек
\newcommand{\ltheadPowerJumpers}{}
\renewcommand{\ltheadPowerJumpers}{
\hline
\multicolumn{1}{m{4.5cm}|}{\centering Имя\-/описание цепи}&
\multicolumn{1}{m{5.5cm}|}{\centering Кон\-троль\-ная точка №1 (вывод компонента)}&
\multicolumn{1}{m{5.5cm}}{\centering Кон\-троль\-ная точка №2 (вывод компонента)}\\
\hline}

\begin{longtable}{m{4.5cm}|m{5.5cm}|m{5.5cm}}
\caption{Точки для установки перемычек по цепям питания}
\label{tab:power_jumper_points}\\
% первая шапка     
\ltheadPowerJumpers
\endfirsthead   
% последующие шапки 
\caption*{\it{Продолжение таблицы} \thetable}\\
\ltheadPowerJumpers
\endhead
% концевики
%\multicolumn{3}{c}{\hidehline}
\endfoot
\endlastfoot
%============ содержимое таблицы==============================
%%Строка
\multicolumn{1}{m{4.5cm}|}{\centering VCC3.3}&
\multicolumn{1}{m{5.5cm}|}{\centering K78}&
\multicolumn{1}{m{5.5cm}}{\centering K109}\\\hline
%%Строка
\multicolumn{1}{m{4.5cm}|}{\centering VCC2.5}&
\multicolumn{1}{m{5.5cm}|}{\centering K122}&
\multicolumn{1}{m{5.5cm}}{\centering K123}\\\hline
%%Строка
\multicolumn{1}{m{4.5cm}|}{\centering VCC1.0}&
\multicolumn{1}{m{5.5cm}|}{\centering K127}&
\multicolumn{1}{m{5.5cm}}{\centering K128}\\\hline
%%Строка
\multicolumn{1}{m{4.5cm}|}{\centering DDR3\_VDD}&
\multicolumn{1}{m{5.5cm}|}{\centering K117}&
\multicolumn{1}{m{5.5cm}}{\centering K116}\\\hline
%%Строка
\multicolumn{1}{m{4.5cm}|}{\centering DDR3\_VTT}&
\multicolumn{1}{m{5.5cm}|}{\centering K115}&
\multicolumn{1}{m{5.5cm}}{\centering K114}\\\hline
%%Строка
\multicolumn{1}{m{4.5cm}|}{\centering DDR3\_VREF}&
\multicolumn{1}{m{5.5cm}|}{\centering K113}&
\multicolumn{1}{m{5.5cm}}{\centering K112}\\\hline
\end{longtable}
      

  \subpoint Проверьте правильность функционирования вторичных источников питания \DocProductShortTitle~ 
	    3,3~В (цепь VCC3.3), 2,5~В (цепь VCC2.5), 1~В (цепь VCC1.0), 1,5~В (цепь DDR3\_VDD), 0,75~В (цепь DDR3\_VTT), 0,75~В (цепь DDR3\_VREF), 1,8~В (цепь +1.8V\_CPLD)
	    под нагрузкой .
  \label {test_ps_with_load} %Начало описания проверки ИП под нагрузкой    
    \subsubpoint Включите источник питания A1.
    \subsubpoint С помощью мультиметра измерьте значение выходного напряжения вторичых источников питания \DocProductShortTitle~ 
		  3,3 В (цепь VCC3.3), 2,5~В (цепь VCC2.5), 1~В (цепь VCC1.0), 1,5~В (цепь DDR3\_VDD), 0,75~В (цепь DDR3\_VTT), 0,75~В (цепь DDR3\_VREF)
		  относительно общей точки схемы (цепь GND). 
		  Измерение проводить между контрольными точками, указанными в таблице \ref{tab:test_v}.        
    \subsubpoint Отключите источник питания A1.
    \subsubpoint Если измеренные напряжения находятся в пределах, указанных в таблице \ref{tab:test_v}, то считается, что вторичные источники питания правильно функционируют под нагрузкой.
    \subsubpoint Если обнаружено несоответствие, то следует определить причину неисправности и устранить её, затем повторить действия, начиная с пункта~\ref{test_ps_with_load}.
  
  \subpoint Отпаяйте адаптер питания.
  \subpoint Произведите визуальный контроль состояния \DocProductShortTitle .

%%---------------------------------------------------------------------------------------------------
  \begin{comment}%Секция временно исключена из инструкции
  
  \pointbold{Проверка осциллограмм основных синалов}
   
   \subpoint В случае несовпадения осциллограмм, определить неисправность
    \begin{table}[h]
    \caption{Контрольные осциллограммы сигналов}
    \label{tab:testpoint_oscillograms}
    \begin{tabular}{l|l|l|c}
    \hline
    \multicolumn{1}{m{3cm}|}{\centering Имя сигнала}&
    \multicolumn{1}{m{3cm}|}{\centering Контрольная точка (вывод компонента)}&
    \multicolumn{1}{m{4,5cm}|}{\centering Описание сигнала}&
    \multicolumn{1}{m{5cm}}{\centering Номинальное Значение (осциллограмма) сигнала}\\\hline
    
    \multicolumn{1}{m{3cm}|}{IFC\_CS[0]\#} &
    \multicolumn{1}{m{3cm}|}{Rxx, вывод ХХ} &
    \multicolumn{1}{m{4,5cm}|}{Сигнал выбора чипа для NOR1} &
    \multicolumn{1}{m{5cm}}{***} \\\hline
    
    \multicolumn{1}{m{3cm}|}{IFC\_CS[1]\#} &
    \multicolumn{1}{m{3cm}|}{Rxx, вывод ХХ} &
    \multicolumn{1}{m{4,5cm}|}{Сигнал выбора чипа для NOR2} &
    \multicolumn{1}{m{5cm}}{***} \\\hline
    
    \end{tabular}
    \end{table}
    
    
    NAND\_CS\#
    NOR\_RST\#    
    TSEC\_MDC K18
    
    XSIG060111 K18 ЧАСТОТА 25 МГЦ TETH
    RESET\_E1
    
    XSIG070270 K28 
    RESET\_E2
    
    XSIG080226 K35
    RESET\_E3
    
    SYSRESET\#
    MASKABLERESET\#
    
    ЧАСТОТА 25 МГЦ ДЛЯ СИНЕЗАТОРА 100МГЦ И ФОРМИРОВАТЕЛЯ REF\_CLK K61
    ВЫХОД 25МГЦ СИНЕЗАТОРА 100МГЦ K62
    
    REF\_CLK\_VPX-/REF\_CLK\_VPX+ R162/R163
    HRESET\# K64
    READY K69
    XSIG120011 66,666 МГЦ K63
    
    C1 DA3:4
    C2 DA3:18
    C3 DA3:3
    C4 DA3:17
    C5 DA3:1
    C6 DA3:20
    CMP\_RST\# K81
    
    \end{comment}
    
%%---------------------------------------------------------------------------------------------------  
  \pointbold{Программирование ПЛИС}
    \subpoint Произведите программирование ПЛИС в соответствии с инструкцией \DocProductSignature~И1.
    
%%---------------------------------------------------------------------------------------------------
  \pointbold{Проверка функционирования в нормальных климатических условиях}
    \subpoint Произведите проверку функционирования \DocProductShortTitle~ в нормальных климатических условиях (НКУ) согласно разделу 3 технических условий \DocModuleSignature~ТУ. 
    \subpoint В случае невыполнения программы проверки, определите и устраните неисправность. 
	      Возможные проявления и причины неисправностей перечислены в разделе~\ref{failure_simptoms_reasons}.
	      %Возможные причины неисправностей перечислены в разделе~\ref{failure_reasons}.
	      
    \begin{footnotesize}    
      Примечание --- При поиске неисправности можно воспользоваться утилитой \emph{fpga}. Порядок использования утилиты описан в разделе ~\ref{sec:fpga_utility_usage}.
    \end{footnotesize}
    
%%---------------------------------------------------------------------------------------------------  
  \pointbold{Влагозащитное и антикоррозионное покрытие}
    \subpoint После устранения неисправностей \DocProductShortTitle~ произведите лакировку в соответствии с требованиями чертежа \DocProductSignature~ СБ.

%%---------------------------------------------------------------------------------------------------
  \pointbold{Технологическая приработка}
    \subpoint После лакировки платы подвергнуть \DocProductShortTitle~ технологической приработке в форме термоциклирования.
    \subpoint Технологическую приработку проводить тремя циклами, следующими непрерывно друг за другом.
    \subpoint Во время технологической приработки \DocProductShortTitle~ находится в обесточенном состоянии.
    \subpoint Скорость изменения температуры во время технологической приработки \DocProductShortTitle не регламентируется.
    \subpoint Последовательность одного цикла:
      \subsubpoint \DocProductShortTitle~ поместить в камеру тепла и холода, температура в которой заранее доведена до минус~(40~$\pm$~3)~\textcelsius~;
      \subsubpoint Выдержать \DocProductShortTitle~ при заданной температуре в течение одного часа; \label{thermo_cycle}      
      \subsubpoint Изменить заданную температуру в камере на плюс~(80~$\pm$~3)~\textcelsius;
      \subsubpoint По достижении температуры в камере заданного значения выдержать \DocProductShortTitle~ при этой температуре в течение одного часа;
      \subsubpoint Если текущий цикл является конечным в последовательности, то извлечь \DocProductShortTitle~ из камеры.
      \subsubpoint Если текущий цикл не является конечным в последовательности, то изменить заданную температуру в камере на минус~(40~$\pm$~3)~\textcelsius~ 
		    и по достижении температуры заданного значения повторить цикл, начиная с пункта~\ref{thermo_cycle}.
      
    \subpoint По окончании третьего цикла \DocProductShortTitle~ извлечь из камеры тепла и холода и выдержать в нормальных климатических условиях в течение одного часа.
    \subpoint Провести внешний осмотр изделия с целью определения сохранности покрытий, отсутствия коррозии и обнаружения других возможных дефектов.

%%---------------------------------------------------------------------------------------------------
   \pointbold{Завершающая проверка функционирования в нормальных климатических условиях}
      \subpoint Произведите проверку функционирования \DocProductShortTitle~ в нормальных климатических условиях (НКУ) согласно разделу 3 технических условий \DocModuleSignature~ТУ.
    