%%---------------------------------------------------------------------------------------------------      
%\section{Возможные проявления неисправностей}
\section{Возможные проявления и причины неисправностей}
\label {failure_simptoms_reasons}
%%%%- Группа Симптомов --------------------------------------------------------------------------------------------------
  \pointbold {Ошибка при выполнении теста <<Проверка загрузки U-boot>>}
    \label{defects:fault_uboot}
    \subpoint Возможные причины данного сбоя:
      \begin{enumerate}	
	\item неверные значения делителей тактовой частоты; неверная последовательность сигналов, запускающих процессор;
	  \begin{enumerate}
	    \item запрограммирован несоответствующий загрузочный код ПЛИС;
	    \item повреждение (дефект монтажа) микросхемы ПЛИС DD30 или её сервисных компонентов;
	  \end{enumerate}
	\item тест ОЗУ, реализованный в загрузчике завершился с ошибками. Повреждение (дефект монтажа) микросхем ОЗУ DD2, DD3, DD4, DD5 или их сервисных компонентов;
	\item процессор не прошел стадию инициализации. Повреждение (дефект монтажа) микросхемы процессора DD1 или её сервисных компонентов.
      \end{enumerate}
    
%%%%- Группа Симптомов --------------------------------------------------------------------------------------------------
  \pointbold {Ошибка при выполнении теста <<Комплексный тест NOR1 и NOR2>>}    
    \subpoint При наличии диагностического сообщения <<Ошибка: Не установлено соединение через технологический Ether\-net>> возможны следующие причины сбоя:
      \begin{enumerate}  
	\item сбой в работе технологического интерфейса Ethernet. Возможные причины описаны в~\ref{defects:fault_teth};
	\item сбой в работе технологического интерфейса Ethernet (на стороне стенда ТСОиК).
      \end{enumerate}
      
    \subpoint При наличии диагностического сообщения <<Ошибка: не удается подключиться к TFTP серверу ТСОиК>> возможны следующие причины сбоя:
      \begin{enumerate}  
	\item на стендовом компьютере ТСОиК не запущена программа TFTP сервера;
	\item программа TFTP сервера не настроена в соответствии с интструкцией по эксплуатации ТСОиК;
	\item в директории TFTP сервера нет двоичных файлов, предназначенных для комплексного теста NOR1 и NOR2.
      \end{enumerate}
      
    \subpoint При наличии диагностического сообщения <<Время ожидания ответа вышло>> возможны следующие причины сбоя:
      \begin{enumerate} 
	\item повреждение (дефект монтажа) РПЗУ NOR1 или его сервисных компонентов;
	\item повреждение (дефект монтажа) РПЗУ NOR2 или его сервисных компонентов;
	\item повреждение (дефект монтажа) процессора DD1 или его сервисных компонентов;
	\item повреждение (дефект монтажа) линий связи локальной шины (между процессором и РПЗУ NOR1);
	\item повреждение (дефект монтажа) линий связи локальной шины (между процессором и РПЗУ NOR2);
	\item сбой в работе технологического интерфейса RS232C.	  
      \end{enumerate}    
      
%%%%- Группа Симптомов --------------------------------------------------------------------------------------------------
  \pointbold {Ошибка при выполнении теста <<Запись Li\-nux OS в NOR1>>}
    \subpoint При наличии диагностического сообщения  <<Ошибка: Не установлено соединение через технологический Ether\-net>>  возможны следующие причины сбоя:
      \begin{enumerate}  
	  \item сбой в работе технологического интерфейса Ether\-net. Возможные причин описаны в~\ref{defects:fault_teth};
	  \item сбой в работе технологического интерфейса Ether\-net (на стороне стенда ТСОиК).
      \end{enumerate}
    
    \subpoint При наличии диагностического сообщения  <<Ошибка: не удается подключиться к TFTP серверу ТСОиК>> возможны следующие причины сбоя:
      \begin{enumerate}  
	\item на стендовом компьютере ТСОиК не запущена программа TFTP сервера;
	\item программа TFTP сервера не настроена в соответствии с интструкцией по эксплуатации ТСОиК;
	\item в директории TFTP сервера нет двоичных файлов, предназначенных для комплексного теста NOR1 и NOR2.
      \end{enumerate}
    
    \subpoint При наличии диагностического сообщения <<Время ожидания ответа вышло>> возможны следующие причины сбоя:
      \begin{enumerate}
	\item сбой в работе технологического интерфейса RS232C.	  
      \end{enumerate}  
    
%%%%- Группа Симптомов --------------------------------------------------------------------------------------------------
  \pointbold {Ошибка при выполнении теста <<Проверка сигнала перезагрузки>>}
    \subpoint  При наличии диагностического сообщения <<Ошибка при тесте самотестирования, перезагрузите плату>> возможны следующие причины сбоя:
      \begin{enumerate}		
	\item причины аналогичны перечисленным в ~\ref{defects:fault_uboot};
      \end{enumerate}
    
    \subpoint При отсутствии диагностических сообщения о сбое возможны следующие его причины:
      \begin{enumerate}
	\item сбой в работе технологического интерфейса RS232C.	  
      \end{enumerate}
      
%%%%- Группа Симптомов --------------------------------------------------------------------------------------------------
  \pointbold {Ошибка при выполнении теста <<Запуск Li\-nux>>}
    \subpoint При наличии диагностического сообщения <<Ошибка: в NOR1 образ ядра поврежден>> возможны следующие причины сбоя:
      \begin{enumerate}
	\item повреждение (дефект монтажа) РПЗУ NOR1 или его сервисных компонентов;
	\item повреждение (дефект монтажа) микросхемы процессора DD1 или его сервисных компонентов;
	\item повреждение (дефект монтажа) линий связи локальной шины (между процессором и РПЗУ NOR1).
      \end{enumerate}
      
    \subpoint При отсутствии диагностических сообщения о сбое возможны следующие его причины:
      \begin{enumerate}
	\item сбой в работе технологического интерфейса RS232C.
      \end{enumerate}
      
%%%%- Группа Симптомов --------------------------------------------------------------------------------------------------
  \pointbold {Ошибка при выполнении теста <<Проверка порта TRS>>}
    \subpoint При отсутствии диагностических сообщения о сбое возможны следующие его причины:
      \begin{enumerate}	
	\item повреждение (дефект монтажа) микросхемы драйвера DD31 или её сервисных компонентов;
	\item повреждение (дефект монтажа) микросхемы процессора DD1 или его сервисных компонентов;
	\item повреждение (дефект монтажа) микросхемы ПЛИС DD30;
	\item повреждение (дефект монтажа) линии DD1 до DD30 UART[0]\_SOUT, UART[0]\_SIN;
	\item повреждение (дефект монтажа) линии DD30 до DD31;
	\item повреждение (дефект монтажа) линии DD31 до XP3 TRS\_T, TRS\_R;
	\item повреждение (дефект монтажа) разъема XP3 (VPX);
      \end{enumerate}
      
%%%%- Группа Симптомов -------------------------------------------------------------------------------------------------- 
  \pointbold {Ошибка при выполнении теста <<Запись тестовых скриптов>>}
    \subpoint При отсутствии диагностических сообщения о сбое возможны следующие его причины:
      \begin{enumerate}
	\item на стендовом компьютере ТСОиК не запущена программа HTTP сервера;
	\item программа HTTP сервера не настроена в соответствии с интструкцией по эксплуатации ТСОиК;
	\item в директории HTTP сервера стендовоко ПК нет файлов, предназначенных для проверки модуля по ТУ;
	\item сбой в работе технологического интерфейса RS232C.
      \end{enumerate} 
      
%%%%- Группа Симптомов --------------------------------------------------------------------------------------------------
  \pointbold {Ошибка при выполнении теста <<Проверка доступа к ПЛИС>>}
    \subpoint При наличии диагностического сообщения <<Тест завершился неудачей>> возможны следующие причины сбоя:
      \begin{enumerate}
	\item сбой в линиях связи процессора с ПЛИС (SPI, GPIO);
	\item повреждение (дефект монтажа) процессора DD1 или его сервисных компонентов;
	\item повреждение (дефект монтажа) микросхемы ПЛИС DD30 или его сервисных компонентов;
	\item запрограммирован несоответствующий загрузочный код ПЛИС.
      \end{enumerate}
      
    \subpoint При наличии диагностического сообщения <<Время ожидания ответа вышло>> возможны следующие причины сбоя:
      \begin{enumerate}	  
	\item cбой в работе технологического интерфейса RS232C.
      \end{enumerate}
      
%%%%- Группа Симптомов --------------------------------------------------------------------------------------------------
  \pointbold {Ошибка при выполнении теста <<Проверка РПЗУ NAND>>}
    \subpoint При наличии диагностического сообщения <<Тест завершился неудачей>> возможны следующие причины сбоя:
      \begin{enumerate}
	\item повреждение (дефект монтажа) РПЗУ NAND DD6 или его сервисных компонентов;
	\item повреждение (дефект монтажа) процессора DD1 или его сервисных компонентов;
	\item повреждение (дефект монтажа) линий связи локальной шины (между процессором и РПЗУ NAND);
	\item повреждение (дефект монтажа) линии NAND\_CS\#.
      \end{enumerate}
      
    \subpoint При наличии диагностического сообщения <<Время ожидания ответа вышло>> возможны следующие причины сбоя:
      \begin{enumerate}
	\item cбой в работе технологического интерфейса RS232C.
      \end{enumerate}   
      
%%%%- Группа Симптомов --------------------------------------------------------------------------------------------------
  \pointbold {Ошибка при выполнении теста <<Проверка ЭЗУ nvRAM №1>>}
    \subpoint При наличии диагностического сообщения <<Тест завершился неудачей>> возможны следующие причины сбоя:
      \begin{enumerate}
	\item повреждение (дефект монтажа) ЭЗУ nvRAM №1 или его сервисных компонентов;
	\item повреждение (дефект монтажа) процессора DD1 или его сервисных компонентов;
	\item повреждение (дефект монтажа) линий SPI (между процессором и ЭЗУ nvRAM №1);
	\item повреждение (дефект монтажа) линии NVRAM1\_CS.
      \end{enumerate}
      
    \subpoint При наличии диагностического сообщения <<Время ожидания ответа вышло>> возможны следующие причины сбоя:
      \begin{enumerate}
	\item cбой в работе технологического интерфейса RS232C.
      \end{enumerate}   
      
%%%%- Группа Симптомов --------------------------------------------------------------------------------------------------
  \pointbold {Ошибка при выполнении теста <<Проверка ЭЗУ nvRAM №2>>}
    \subpoint При наличии диагностического сообщения <<Тест завершился неудачей>> возможны следующие причины сбоя:
      \begin{enumerate}
	\item повреждение (дефект монтажа) ЭЗУ nvRAM №2 или его сервисных компонентов;
	\item повреждение (дефект монтажа) процессора DD1 или его сервисных компонентов;
	\item повреждение (дефект монтажа) линий SPI (между процессором и ЭЗУ nvRAM №2);
	\item повреждение (дефект монтажа) линии NVRAM2\_CS.
      \end{enumerate}
      
    \subpoint При наличии диагностического сообщения <<Время ожидания ответа вышло>> возможны следующие причины сбоя:
      \begin{enumerate}
	\item cбой в работе технологического интерфейса RS232C.
      \end{enumerate}   
      
%%%%- Группа Симптомов --------------------------------------------------------------------------------------------------
  \begin{sloppypar}
  
  \pointbold {Ошибка при выполнении теста <<Проверка РПЗУ I2C EEPROM>>}
    \subpoint При наличии диагностического сообщения <<Тест завершился неудачей>> возможны следующие причины сбоя:
      \begin{enumerate}
	\item повреждение (дефект монтажа) РПЗУ I2C EEPROM или его сервисных компонентов;
	\item повреждение (дефект монтажа) буфера I2C DD18 или его сервисных компонентов;
	\item повреждение (дефект монтажа) процессора DD1 или его сервисных компонентов;
	\item повреждение (дефект монтажа) линий шины IIC[1] (между процессором и буфером);
	\item повреждение (дефект монтажа) линий шины IIC\_INT (между I2C EEPROM и буфером);
	\item повреждение (дефект монтажа) лини EN\_IIC\_INT.
      \end{enumerate}
      
    \subpoint При наличии диагностического сообщения <<Время ожидания ответа вышло>> возможны следующие причины сбоя:
      \begin{enumerate}
	\item cбой в работе технологического интерфейса RS232C.
      \end{enumerate}
  \end{sloppypar}
  
%%%%- Группа Симптомов --------------------------------------------------------------------------------------------------
  \pointbold {Ошибка при выполнении теста <<Проверка I2C термодатчиков>>}
    \subpoint При наличии диагностического сообщения <<Тест завершился неудачей>> возможны следующие причины сбоя:
      \begin{enumerate}
	\item повреждение (дефект монтажа) термодатчиков DD14,DD15 или их сервисных компонентов;
	\item повреждение (дефект монтажа) буфера I2C DD18 или его сервисных компонентов;
	\item повреждение (дефект монтажа) процессора DD1 или его сервисных компонентов;
	\item повреждение (дефект монтажа) линий шины IIC[1] (между процессором и буфером);
	\item повреждение (дефект монтажа) линий шины IIC\_INT (между термодатчиками и буфером);
	\item повреждение (дефект монтажа) лини EN\_IIC\_INT.
      \end{enumerate}
      
    \subpoint При наличии диагностического сообщения <<Время ожидания ответа вышло>> возможны следующие причины сбоя:
      \begin{enumerate}
	\item cбой в работе технологического интерфейса RS232C.
      \end{enumerate}
      
%%%%- Группа Симптомов --------------------------------------------------------------------------------------------------
  \pointbold {Ошибка при выполнении теста <<Проверка порта Ethernet (TEth)>>}
    \label{defects:fault_teth}
    \subpoint При наличии диагностического сообщения <<Тест завершился неудачей>> возможны следующие причины сбоя:      
      \begin{enumerate}\begin{sloppypar}
	\item повреждение (дефект монтажа) микросхемы DD10 (88E1111) или её сервисных компонентов (G2, T1, VD1 и так далее). Дополнительные дифференцирующие признаки:
	  \begin{enumerate}
	    \item отсутствие синусоидального напряжения с частотой 25 МГц на выходе J9 (XTAL2), тестовая точка K22;
	    \item несоответствие сопротивления входных и выходных каскадов интерфейса SGMII номиналу (см. таблицу \ref{tab:sgmii_r});
	  \end{enumerate}
	    %% Таблица со значениями сопротивлений
  \newcommand{\ltheadsgmiiR}{}
  \renewcommand{\ltheadsgmiiR}{
    \hline
    \multicolumn{1}{m{8.0cm}|}{\centering Имя\-/описание цепи}&
    \multicolumn{1}{m{2.5cm}|}{\centering Измеряемая микросхема}&
    \multicolumn{1}{m{2.9cm}|}{\centering Ми\-ни\-маль\-ное зна\-че\-ние со\-про\-тив\-ле\-ния, Ом}&
    \multicolumn{1}{m{2.9cm} }{\centering Мак\-си\-маль\-ное зна\-че\-ние со\-про\-тив\-ле\-ния, Ом}\\
    \hline}  
  \begin{longtable}{m{8.0cm}|m{2.5cm}|m{2.9cm}|m{2.9cm}}
    \caption{Сопротивления цепей питания}
    \label{tab:sgmii_r}\\
    % первая шапка     
    \ltheadsgmiiR
    \endfirsthead   
    % последующие шапки 
    \caption*{\it{Продолжение таблицы} \thetable}\\
    \ltheadsgmiiR
    \endhead
    % концевики
    %\multicolumn{3}{c}{\hidehline}
    \endfoot
    \endlastfoot
    %============ содержимое таблицы==============================
    %%Строка
    \multicolumn{1}{m{8.0cm}|}{\centering SGMII1\_TX+, SGMII1\_TX-, SGMII2\_TX+, SGMII2\_TX-, SGMII3\_TX+, SGMII3\_TX-}&
    \multicolumn{1}{m{2.5cm}|}{\centering DD1 (P1010)}&
    \multicolumn{1}{m{2.9cm}|}{\centering 190 }&
    \multicolumn{1}{m{2.9cm} }{\centering 210}\\\hline
     %%Строка
    \multicolumn{1}{m{8.0cm}|}{\centering SGMII1\_TX\_C+, SGMII1\_TX\_C-, SGMII2\_TX\_C+, SGMII2\_TX\_C-, SGMII3\_TX\_C+, SGMII3\_TX\_C-}&
    \multicolumn{1}{m{2.5cm}|}{\centering DD10, DD11, DD12 (88E1111)}&
    \multicolumn{1}{m{2.9cm}|}{\centering 1700 }&
    \multicolumn{1}{m{2.9cm} }{\centering 2300}\\\hline
     %%Строка
    \multicolumn{1}{m{8.0cm}|}{\centering SGMII1\_RX+, SGMII1\_RX-, SGMII2\_RX+, SGMII2\_RX-, SGMII3\_RX+, SGMII3\_RX-}&
    \multicolumn{1}{m{2.5cm}|}{\centering DD1 (P1010)}&
    \multicolumn{1}{m{2.9cm}|}{\centering 2800}&
    \multicolumn{1}{m{2.9cm} }{\centering 3200}\\\hline
     %%Строка
    \multicolumn{1}{m{8.0cm}|}{\centering SGMII1\_RX\_C+, SGMII1\_RX\_C-, SGMII2\_RX\_C+, SGMII2\_RX\_C-, SGMII3\_RX\_C+, SGMII3\_RX\_C-}&
    \multicolumn{1}{m{2.5cm}|}{\centering DD10, DD11, DD12 (88E1111)}&
    \multicolumn{1}{m{2.9cm}|}{\centering 1900 }&
    \multicolumn{1}{m{2.9cm} }{\centering 2100}\\\hline
  \end{longtable}
  %%---------------------------------------------------------------------------------------------------  
	  
	\item повреждение (дефект монтажа) микросхемы DD1(P1010) или её сервисных компонентов. Дополнительные дифференцирующие признаки:
	  \begin{enumerate}	  
	    \item несоответствие сопротивления входных и выходных каскадов интерфейса SGMII номиналу (см. таблицу \ref{tab:sgmii_r});    
	  \end{enumerate}	  
	\item повреждение (дефект монтажа) линий связи микросхемы DD10 (88E1111) с микросхемой DD30 (5M2210);
	\item повреждение (дефект монтажа) линий связи микросхемы DD12 (88E1111) с микросхемой DD30 (5M2210) с учётом развязывающих конденсаторов С99, С105, С111, С114;
	\item отсутствие сигнала синхронизации по линии SD1\_REFCLK+/SD1\_REFCLK- (компоненты G5, DA2, DA29 и их сервисные компоненты);
	\item повреждение стендового порта Ethernet.
      \end{sloppypar}\end{enumerate}
      
    \subpoint При наличии диагностического сообщения <<Время ожидания ответа вышло>> возможны следующие причины сбоя:
      \begin{enumerate}
	\item cбой в работе технологического интерфейса RS232C.
      \end{enumerate}
      
%%%%- Группа Симптомов --------------------------------------------------------------------------------------------------
  \pointbold {Ошибка при выполнении теста <<Проверка порта Ethernet AFDX1>>}
    \subpoint При наличии диагностического сообщения <<Тест завершился неудачей>> возможны следующие причины сбоя:
      \begin{enumerate}\begin{sloppypar}
	\item повреждение (дефект монтажа) микросхемы DD11 (88E1111) или её сервисных компонентов (G3, T2, VD2 и так далее). Дополнительные дифференцирующие признаки:
	  \begin{enumerate}
	    \item отсутствие синусоидального напряжения с частотой 25 МГц на выходе J9 (XTAL2), тестовая точка K29;
	    \item несоответствие сопротивления входных и выходных каскадов интерфейса SGMII номиналу (см. таблицу \ref{tab:sgmii_r});    
	  \end{enumerate}
	\item повреждение (дефект монтажа) микросхемы DD1(P1010) или её сервисных компонентов. Дополнительные дифференцирующие признаки:
	  \begin{enumerate}	  
	    \item несоответствие сопротивления входных и выходных каскадов интерфейса SGMII номиналу (см. таблицу \ref{tab:sgmii_r}); 
	  \end{enumerate}
	\item повреждение (дефект монтажа) линий связи микросхемы DD11 (88E1111) с микросхемой DD30 (5M2210);
	\item повреждение (дефект монтажа) линий связи микросхемы DD12 (88E1111) с микросхемой DD30 (5M2210) с учётом развязывающих конденсаторов С100, С106, С149, С152;
	\item отсутствие сигнала синхронизации по линии SD1\_REFCLK+/SD1\_REFCLK- (компоненты G5, DA2, DA29 и их сервисные компоненты);
	\item повреждение стендового порта Ethernet.
      \end{sloppypar}\end{enumerate}
      
    \subpoint При наличии диагностического сообщения <<Время ожидания ответа вышло>> возможны следующие причины сбоя:
      \begin{enumerate}
	\item cбой в работе технологического интерфейса RS232C.
      \end{enumerate}
      
%%%%- Группа Симптомов --------------------------------------------------------------------------------------------------
  \pointbold {Ошибка при выполнении теста <<Проверка порта Ethernet AFDX2>>}
    \subpoint При наличии диагностического сообщения <<Тест завершился неудачей>> возможны следующие причины сбоя:
      \begin{enumerate}\begin{sloppypar}
	\item повреждение (дефект монтажа) микросхемы DD12 (88E1111) или её сервисных компонентов (G4, T3, VD3 и так далее). Дополнительные дифференцирующие признаки:
	  \begin{enumerate}
	    \item отсутствие синусоидального напряжения с частотой 25 МГц на выходе J9 (XTAL2), тестовая точка K36.
	    \item несоответствие сопротивления входных и выходных каскадов интерфейса SGMII номиналу (см. таблицу \ref{tab:sgmii_r});
	  \end{enumerate}
	\item повреждение (дефект монтажа) микросхемы DD1(P1010) или её сервисных компонентов. Дополнительные дифференцирующие признаки:
	  \begin{enumerate}	  
	    \item несоответствие сопротивления входных и выходных каскадов интерфейса SGMII номиналу (см. таблицу \ref{tab:sgmii_r});
	  \end{enumerate}
	\item повреждение (дефект монтажа) линий связи микросхемы DD12 (88E1111) с микросхемой DD30 (5M2210) с учётом развязывающих конденсаторов С101, С107, С187, С190;
	\item повреждение (дефект монтажа) линий связи микросхемы DD12 (88E1111) с микросхемой DD1(P1010);
	\item отсутствие сигнала синхронизации по линии SD1\_REFCLK+/SD1\_REFCLK- (компоненты G5, DA2, DA29 и их сервисные компоненты);
	\item повреждение стендового порта Ethernet.
      \end{sloppypar}\end{enumerate}
      
    \subpoint При наличии диагностического сообщения <<Время ожидания ответа вышло>> возможны следующие причины сбоя:
      \begin{enumerate}
	\item cбой в работе технологического интерфейса RS232C.
      \end{enumerate}
      
%%%%- Группа Симптомов --------------------------------------------------------------------------------------------------
  \begin{sloppypar}
  \pointbold {Ошибка при выполнении теста <<Проверка порта PCIe\_VPX>>}
  \label{defects:fault_pcie_vpx}
    \subpoint При наличии диагностического сообщения <<Тест завершился неудачей>> возможны следующие причины сбоя:
      \begin{enumerate}
	\item повреждение (дефект монтажа) микросхемы DD1(P1010) или её сервисных компонентов;
	\item повреждение (дефект монтажа) линий связи микросхемы DD1(P1010) с разъемом XP3 (VPX) с учётом развязывающих конденсаторов С103, С109, проходных резисторов R200, R201;
	\item повреждение (дефект монтажа) разъема XP3 (VPX);
	\item плохой контакт разъёма (разъёмов) стендового мезонина с разъёмом (разъёмами) \DocProductShortTitle;
	\item при наличии в тестах сбоя типа ~\ref{defects:fault_xmc_sig_diff}, ~\ref{defects:fault_pcie_xmc} (тесты используют общую аппаратуру для проверки) у всех трёх сбоев может быть общий источник;
	\item повреждение стендового устройства PCIe (сетевая карта на модуле МКИМ).
      \end{enumerate}
      
    \subpoint При наличии диагностического сообщения <<Время ожидания ответа вышло>> возможны следующие причины сбоя:
      \begin{enumerate}
	\item cбой в работе технологического интерфейса RS232C.
      \end{enumerate}
   \end{sloppypar}
   
%%%%- Группа Симптомов --------------------------------------------------------------------------------------------------
  \begin{sloppypar}
  \pointbold {Ошибка при выполнении теста <<Проверка порта I2C\_VPX>>}
    \subpoint При наличии диагностического сообщения <<Тест завершился неудачей>> возможны следующие причины сбоя:
      \begin{enumerate}	  
	\item повреждение (дефект монтажа) буфера I2C DD17 или его сервисных компонентов;
	\item повреждение (дефект монтажа) процессора DD1 или его сервисных компонентов;
	\item повреждение (дефект монтажа) линий шины IIC[2] (между процессором и буфером);
	\item повреждение (дефект монтажа) линий шины SM0, SM1 (между буфером и разъёмом);
	\item повреждение (дефект монтажа) лини EN\_IIC\_VPX;
	\item повреждение (дефект монтажа) разъема XP3 (VPX).
      \end{enumerate}
      
    \subpoint При наличии диагностического сообщения <<Время ожидания ответа вышло>> возможны следующие причины сбоя:
      \begin{enumerate}
	\item cбой в работе технологического интерфейса RS232C.
      \end{enumerate}
       
   \end{sloppypar}

%%%%- Группа Симптомов --------------------------------------------------------------------------------------------------
  \begin{sloppypar}
  \pointbold {Ошибка при выполнении теста <<Проверка порта PCIe\_XMC>>}
  \label{defects:fault_pcie_xmc}
    \subpoint При наличии диагностического сообщения <<Тест завершился неудачей>> возможны следующие причины сбоя:
      \begin{enumerate}
	\item повреждение (дефект монтажа) микросхемы DD1(P1010) или её сервисных компонентов;
	\item повреждение (дефект монтажа) линий связи микросхемы DD1(P1010) с разъемом XS1 (XMC) с учётом развязывающих конденсаторов С102, С108, проходных резисторов R205, R206;
	\item повреждение (дефект монтажа) разъема XS2 (XMC);
	\item плохой контакт разъёма (разъёмов) стендового мезонина с разъёмом (разъёмами) \DocProductShortTitle;
	\item при наличии в тестах сбоя типа ~\ref{defects:fault_xmc_sig_diff}, ~\ref{defects:fault_pcie_vpx} (тесты используют общую аппаратуру для проверки) у всех трёх сбоев может быть общий источник;
	\item повреждение стендового устройства PCIe (сетевая карта на модуле МКИМ).
      \end{enumerate}
      
    \subpoint При наличии диагностического сообщения <<Время ожидания ответа вышло>> возможны следующие причины сбоя:
      \begin{enumerate}
	\item cбой в работе технологического интерфейса RS232C.
      \end{enumerate}
    
    \end{sloppypar}
      
%%%%- Группа Симптомов --------------------------------------------------------------------------------------------------
  \begin{sloppypar}
  \pointbold {Ошибка при выполнении теста <<Проверка порта I2C\_XMC>>}
    \subpoint При наличии диагностического сообщения <<Тест завершился неудачей>> возможны следующие причины сбоя:
      \begin{enumerate}	  
	\item повреждение (дефект монтажа) буфера I2C DD16 или его сервисных компонентов;
	\item повреждение (дефект монтажа) процессора DD1 или его сервисных компонентов;
	\item повреждение (дефект монтажа) линий шины IIC[1] (между процессором и буфером);
	\item повреждение (дефект монтажа) линий шины MSCL, MSDA (между буфером и разъёмом);
	\item повреждение (дефект монтажа) лини EN\_IIC\_XMC;
	\item повреждение (дефект монтажа) разъема XS1 (XMC).
      \end{enumerate}
      
    \subpoint При наличии диагностического сообщения <<Время ожидания ответа вышло>> возможны следующие причины сбоя:
      \begin{enumerate}
	\item cбой в работе технологического интерфейса RS232C.
      \end{enumerate} 
    
    \end{sloppypar}
    
%%%%- Группа Симптомов --------------------------------------------------------------------------------------------------
  \pointbold {Ошибка при выполнении теста <<Проверка порта RS232C\_1>>}
    \subpoint При наличии диагностического сообщения <<Ошибка передачи данных, где передающая сторона ВИМ-3U-3>> возможны следующие причины сбоя:
      \begin{enumerate}
	\item повреждение (дефект монтажа) микросхемы драйвера DD32 или её сервисных компонентов;
	\item повреждение (дефект монтажа) процессора DD1 или его сервисных компонентов;
	\item повреждение (дефект монтажа) линий UART[2]\_SOUT;
	\item повреждение (дефект монтажа) линий RS\_T[1];
	\item повреждение (дефект монтажа) лини RS\_DX232;
	\item повреждение (дефект монтажа) разъема XP3 (VPX).
      \end{enumerate}
      
    \subpoint При наличии диагностического сообщения <<Ошибка передачи данных, где принимающая сторона ВИМ-3U-3>> возможны следующие причины сбоя:
      \begin{enumerate}
	\item повреждение (дефект монтажа) микросхемы драйвера DD32 или её сервисных компонентов;
	\item повреждение (дефект монтажа) процессора DD1 или его сервисных компонентов;
	\item повреждение (дефект монтажа) линий UART[2]\_SIN;
	\item повреждение (дефект монтажа) линий RS\_R[1];
	\item повреждение (дефект монтажа) лини RS\_RX232;
	\item повреждение (дефект монтажа) разъема XP3 (VPX).	  
      \end{enumerate}   
      
%%%%- Группа Симптомов --------------------------------------------------------------------------------------------------
  \pointbold {Ошибка при выполнении теста <<Проверка порта RS232C\_2>>}
    \subpoint При наличии диагностического сообщения <<Ошибка передачи данных, где передающая сторона ВИМ-3U-3>> возможны следующие причины сбоя:
      \begin{enumerate}
	\item повреждение (дефект монтажа) микросхемы драйвера DD32 или её сервисных компонентов;
	\item повреждение (дефект монтажа) процессора DD1 или его сервисных компонентов;
	\item повреждение (дефект монтажа) линий UART[3]\_SOUT;
	\item повреждение (дефект монтажа) линий RS\_T[2];
	\item повреждение (дефект монтажа) лини RS\_DX232;
	\item повреждение (дефект монтажа) разъема XP3 (VPX).
      \end{enumerate}
      
    \subpoint При наличии диагностического сообщения <<Ошибка передачи данных, где принимающая сторона ВИМ-3U-3>> возможны следующие причины сбоя:
      \begin{enumerate}
	\item повреждение (дефект монтажа) микросхемы драйвера DD32 или её сервисных компонентов;
	\item повреждение (дефект монтажа) процессора DD1 или его сервисных компонентов;
	\item повреждение (дефект монтажа) линий UART[3]\_SIN;
	\item повреждение (дефект монтажа) линий RS\_R[2];
	\item повреждение (дефект монтажа) лини RS\_RX232;
	\item повреждение (дефект монтажа) разъема XP3 (VPX).
      \end{enumerate}  
      
%%%%- Группа Симптомов --------------------------------------------------------------------------------------------------
  \pointbold {Ошибка при выполнении теста <<Проверка порта RS485>>}
    \subpoint При наличии диагностического сообщения <<Ошибка передачи данных, где передающая сторона ВИМ-3U-3>> возможны следующие причины сбоя:
      \begin{enumerate}
	\item повреждение (дефект монтажа) микросхемы драйвера DD31 или её сервисных компонентов;
	\item повреждение (дефект монтажа) микросхемы процессора DD1 или его сервисных компонентов;
	\item повреждение (дефект монтажа) микросхемы ПЛИС DD30;
	\item повреждение (дефект монтажа) линии TSOUT;
	\item повреждение (дефект монтажа) линии UART[3]\_SOUT;
	\item повреждение (дефект монтажа) линии TRS\_T;
	\item повреждение (дефект монтажа) линий, связывающих DD31 и DD30; 
	\item повреждение (дефект монтажа) разъема XP3 (VPX).
      \end{enumerate}
      
    \subpoint При наличии диагностического сообщения <<Ошибка передачи данных, где принимающая сторона ВИМ-3U-3>> возможны следующие причины сбоя:
      \begin{enumerate}
	\item повреждение (дефект монтажа) микросхемы драйвера DD32 или её сервисных компонентов;
	\item повреждение (дефект монтажа) процессора DD1 или его сервисных компонентов;
	\item повреждение (дефект монтажа) линий UART[3]\_SIN;
	\item повреждение (дефект монтажа) линий шины RS\_R[2];
	\item повреждение (дефект монтажа) лини RS\_RX232;
	\item повреждение (дефект монтажа) разъема XP3 (VPX).
      \end{enumerate} 
      
%%%%- Группа Симптомов --------------------------------------------------------------------------------------------------
  \pointbold {Ошибка при выполнении теста <<Сигналы XMC, сквозные дифференциальные. Прохождение 125 МГц>>}
  \label{defects:fault_xmc_sig_diff}
    \subpoint При отсутствии диагностических сообщения о сбое возможны следующие его причины:
      \begin{enumerate}
	\item повреждение (дефект монтажа) разъема XS2 (XMC);
	\item повреждение (дефект монтажа) разъема XP3 (VPX);
	\item повреждение (дефект монтажа) линий, связывающих XS2 и XP3 (JN16-A[1,3,5...19]. JN16-B[1,3,5...19], JN16-D[1,3,5...19], JN16-E[1,3,5...19]);
	\item плохой контакт разъёма (разъёмов) стендового мезонина с разъёмом (разъёмами) \DocProductShortTitle;
	\item при наличии в тестах сбоя типа ~\ref{defects:fault_pcie_xmc}, ~\ref{defects:fault_pcie_vpx} (тесты используют общую аппаратуру для проверки) у всех трёх сбоев может быть общий источник.
      \end{enumerate}
      
    \subpoint При наличии диагностического сообщения <<Время ожидания ответа вышло>> возможны следующие причины сбоя:
      \begin{enumerate}
	\item cбой в работе технологического интерфейса RS232C.
      \end{enumerate} 
      
%%%%- Группа Симптомов --------------------------------------------------------------------------------------------------
  \begin{sloppypar}  
  \pointbold {Ошибка при выполнении теста <<РК 0В/Обрыв. (DS(0)..DS(9)) Срабатывание>>}
    \subpoint При наличии диагностического сообщения <<РК * - не может быть активирована>> 
	      (где <<*>> - номер разовой команды в диапазоне \{0..9\}) возможны следующие причины сбоя:
      \begin{enumerate}
	\item повреждение (дефект монтажа) компонентов каскада разовой команды № <<*>> (см. \DocProductSignature~Э3);
	\item повреждение (дефект монтажа) предохранителя по цепи питания разовых команд модуля FU1;
	\item повреждение (дефект монтажа) микросхемы ПЛИС DD30;
	\item повреждение (дефект монтажа) разъема XP3 (VPX);
      \end{enumerate}
      
      \subpoint При наличии диагностического сообщения <<РК * и ** возможно замкнуты>> 
		(где <<*>> - номер разовой команды в диапазоне \{0..9\}, <<**>> - номер разовой команды в диапазоне \{0..9\, 15..18, 19..21\}) возможны следующие причины сбоя:
      \begin{enumerate}
	\item повреждение (дефект монтажа) компонентов каскадов разовых команд № <<*>> и №<<**>>, 
	      приводящий к замыканию отдельных цепей, участвующих в их формировании (см. \DocProductSignature~Э3);
	\item повреждение (дефект) печатной платы, приводящий к замыканию отдельных цепей, участвующих в формировании каналов №<<*>> и №<<**>>;
	\item замыкание контактов микросхемы ПЛИС DD30;
	\item повреждение (дефект монтажа) разъема XP3 (VPX), приводящее к замыканию разовых команд № <<*>> и №<<**>>.
      \end{enumerate}
      
  \end{sloppypar}
  
%%%%- Группа Симптомов --------------------------------------------------------------------------------------------------
  \begin{sloppypar}
  \pointbold {Ошибка при выполнении теста <<РК 0В/Обрыв. (DS(15)..DS(22)). Срабатывание>>}
    \subpoint При наличии диагностического сообщения <<РК * - не может быть активирована>>
    	      (где <<*>> - номер разовой команды в диапазоне \{0..9\}) возможны следующие причины сбоя:
      \begin{enumerate}
	\item повреждение (дефект монтажа) компонентов каскада разовой команды № <<*>>, образующих верхнее плечо, 
	      то есть подключаемые через оптореле к цепи 27VF (см. \DocProductSignature~Э3);
	\item повреждение (дефект монтажа) предохранителя по цепи питания разовых команд модуля FU1;
	\item повреждение (дефект монтажа) оптореле V4 (для РК \{15..18\}) или V7 (для РК \{19..21\}) или их сервисных компонентов;
	\item повреждение (дефект монтажа) микросхемы ПЛИС DD30;
	\item повреждение (дефект монтажа) разъема XP3 (VPX).
      \end{enumerate}
      
      \subpoint При наличии диагностического сообщения <<РК * - не исправна (активен IN0P и IN0N)>> 
      	      (где <<*>> - номер разовой команды в диапазоне \{0..9\}) возможны следующие причины сбоя:
      \begin{enumerate}
	\item повреждение (дефект монтажа) компонентов каскада разовой команды № <<*>>, образующих нижнее плечо, 
	      то есть подключаемые через оптореле к цепи 0/27 B (см. \DocProductSignature~Э3);
	\item повреждение (дефект монтажа) оптореле V4 (для РК \{15..18\}) или V7 (для РК \{19..21\}) или их сервисных компонентов;
	\item повреждение (дефект монтажа) микросхемы ПЛИС DD30.
      \end{enumerate}
      
      \subpoint При наличии диагностического сообщения <<РК * и ** сработали одновременно>> возможны следующие причины сбоя:
	      (где <<*>> - номер разовой команды в диапазоне \{0..9\}, <<**>> - номер разовой команды в диапазоне \{0..9\, 15..18, 19..21\}) возможны следующие причины сбоя:
      \begin{enumerate}
	\item повреждение (дефект монтажа) компонентов каскадов разовых команд № <<*>> и №<<**>>, 
	      приводящий к замыканию отдельных цепей, участвующих в их формировании (см. \DocProductSignature~Э3);
	\item повреждение (дефект) печатной платы, приводящий к замыканию отдельных цепей, участвующих в формировании каналов №<<*>> и №<<**>>;
	\item замыкание контактов микросхемы ПЛИС DD30;
	\item повреждение (дефект монтажа) разъема XP3 (VPX), приводящее к замыканию разовых команд № <<*>> и №<<**>>.
      \end{enumerate}
      
   \end{sloppypar}
   
%%%%- Группа Симптомов --------------------------------------------------------------------------------------------------
  \begin{sloppypar}
  \pointbold {Ошибка при выполнении теста <<РК 27В/Обрыв. (DS(15)..DS(22)). Срабатывание>>}
    
   \subpoint При наличии диагностического сообщения <<РК * - не может быть активирована>> 
	    (где <<*>> - номер разовой команды в диапазоне \{0..9\}) возможны следующие причины сбоя:
      \begin{enumerate}
	\item повреждение (дефект монтажа) компонентов каскада разовой команды № <<*>>, образующих нижнее плечо, 
	      то есть подключаемые через оптореле к цепи 0/27 B (см. \DocProductSignature~Э3);
	\item повреждение (дефект монтажа) оптореле V4 (для РК \{15..18\}) или V7 (для РК \{19..21\}) или их сервисных компонентов;
	\item повреждение (дефект монтажа) микросхемы ПЛИС DD30;
	\item повреждение (дефект монтажа) разъема XP3 (VPX).
      \end{enumerate}
      
      \subpoint При наличии диагностического сообщения <<РК * - не исправна (активен IN0P и IN0N)>> 
		(где <<*>> - номер разовой команды в диапазоне \{0..9\}) возможны следующие причины сбоя:
      \begin{enumerate}
	\item повреждение (дефект монтажа) компонентов каскада разовой команды № <<*>>, образующих верхнее плечо, 
	      то есть подключаемые через оптореле к цепи 27VF (см. \DocProductSignature~Э3);
	\item повреждение (дефект монтажа) оптореле V4 (для РК \{15..18\}) или V7 (для РК \{19..21\}) или их сервисных компонентов;
	\item повреждение (дефект монтажа) микросхемы ПЛИС DD30.
      \end{enumerate}
      
      \subpoint При наличии диагностического сообщения <<РК * и ** сработали одновременно>> 
		(где <<*>> - номер разовой команды в диапазоне \{0..9\}, <<**>> - номер разовой команды в диапазоне \{0..9\, 15..18, 19..21\}) возможны следующие причины сбоя:
      \begin{enumerate}
	\item повреждение (дефект монтажа) компонентов каскадов разовых команд № <<*>> и №<<**>>, 
	      приводящий к замыканию отдельных цепей, участвующих в их формировании (см. \DocProductSignature~Э3);
	\item повреждение (дефект) печатной платы, приводящий к замыканию отдельных цепей, участвующих в формировании каналов №<<*>> и №<<**>>;
	\item замыкание контактов микросхемы ПЛИС DD30;
	\item повреждение (дефект монтажа) разъема XP3 (VPX), приводящее к замыканию разовых команд № <<*>> и №<<**>>.
      \end{enumerate}  
    
  \end{sloppypar}
    
%%%%- Группа Симптомов --------------------------------------------------------------------------------------------------
  \begin{sloppypar}
  \pointbold {Ошибка при выполнении теста <<РК 0В/Обрыв. Выходы.  Срабатывание>>}
    \subpoint При наличии диагностического сообщения <<РК * - Ошибка цепи ВСК resh>>
	      (где <<*>> - номер разовой команды в диапазоне \{23..24\}) возможны следующие причины сбоя (отсутствия срабатывания датчика DS[*]RESH):
	\label{subdefects:rk_out_0v_trig_fault_resh}
	\begin{enumerate}
	  \item повреждение (дефект монтажа) компонентов каскада выходной разовой команды № <<*>>, образующих нижнее плечо, 
		то есть транзисторы и сервисные компоненты, подключающие цепь 0/27 B (см. \DocProductSignature~Э3);
	  \item повреждение (дефект монтажа) компонентов датчиков, формирующих сигналы DS[*]RESH или их сервисных компонентов, приводящее к отсутствию срабатывания датчика DS[*]RESH;
	  \item повреждение (дефект монтажа) микросхемы ПЛИС DD30;
	  \item повреждение (дефект монтажа) стабилитрона VD22, C420, C421, формирующих напряжение 8 В;
	  \item повреждение (дефект монтажа) предохранителя по цепи питания разовых команд модуля FU1.
	\end{enumerate} 
	
    \subpoint При наличии диагностического сообщения <<РК * - Ошибка цепи ВСК resl>> 
	      (где <<*>> - номер разовой команды в диапазоне \{23..24\}) возможны следующие причины сбоя (ложного срабатывания датчика DS[*]RESL):
	 \label{subdefects:rk_out_0v_trig_fault_resl}
	 \begin{enumerate}
	  \item повреждение (дефект монтажа) компонентов каскада выходной разовой команды № <<*>>, образующих верхнее плечо, 
		то есть транзисторы и сервисные компоненты, подключающие цепь 27VF (см. \DocProductSignature~Э3);
	  \item повреждение (дефект монтажа) компонентов датчиков, формирующих сигналы DS[*]RESL или их сервисных компонентов, приводящее к ложному срабатыванию датчика DS[*]RESL;
	  \item повреждение (дефект монтажа) микросхемы ПЛИС DD30.
	\end{enumerate}
	
    \subpoint При наличии диагностического сообщения <<РК * - Ошибка цепи ВСК resh (проверка отсутствия замыкания)>> 
	      (где <<*>> - номер разовой команды в диапазоне \{23..24\}) возможны следующие причины сбоя (ложного срабатывания датчика DS[*]RESH, относящегося к каналу, отличному от проверяемого):
	  \label{subdefects:rk_out_0v_trig_fault_resh_cross}	  
	  \begin{enumerate}
	    \item повреждение (дефект монтажа) компонентов каскадов разовых команд №23 и №24, 
		  приводящий к замыканию отдельных цепей, участвующих в формировании РК №<<*>>(см. \DocProductSignature~Э3); 
	    \item повреждение (дефект) печатной платы, приводящий к замыканию отдельных цепей, участвующих в формировании каналов №23 и №24;
	    \item замыкание контактов микросхемы ПЛИС DD30;
	    \item повреждение (дефект монтажа) разъема XP3 (VPX), приводящее к замыканию разовых команд №23 и №24.
	  \end{enumerate}
      
    \subpoint При наличии диагностического сообщения <<РК * - Ошибка цепи ВСК resl (проверка отсутствия замыкания)>> 
	      (где <<*>> - номер разовой команды в диапазоне \{23..24\}) возможны следующие причины сбоя (ложного срабатывания датчика DS[*]RESL, относящегося к каналу, отличному от проверяемого):
	  \label{subdefects:rk_out_0v_trig_fault_resl_cross}
	  \begin{enumerate}	     
	    \item повреждение (дефект монтажа) компонентов каскадов разовых команд №23 и №24, 
		  приводящий к замыканию отдельных цепей, участвующих в формировании РК №<<*>>(см. \DocProductSignature~Э3);
	    \item замыкание контактов микросхемы ПЛИС DD30.
	  \end{enumerate}
    
    \subpoint При наличии диагностического сообщения <<РК * - Ошибка цепи ds\_r\_h МКИ>> 
	      (где <<*>> - номер разовой команды в диапазоне \{23..24\}) возможны следующие причины сбоя:    
	  \begin{enumerate}
	    \item причины аналогичны перечисленным в ~\ref{subdefects:rk_out_0v_trig_fault_resl};%Именно, ds_r_h соответствует resl!
	    \item повреждение (дефект монтажа) разъема XP3 (VPX).
	  \end{enumerate}
    
    \subpoint При наличии диагностического сообщения <<РК * - Ошибка цепи ds\_r\_l МКИ>> 
	      (где <<*>> - номер разовой команды в диапазоне \{23..24\}) возможны следующие причины сбоя:
	  \begin{enumerate}
	    \item причины аналогичны перечисленным в ~\ref{subdefects:rk_out_0v_trig_fault_resh};%Именно, ds_r_l соответствует resh!
	    \item повреждение (дефект монтажа) разъема XP3 (VPX).
	  \end{enumerate}

%%-- Эти пункты сложно объяснить. Ошибка может возникать при наличии смежной ошибки и означает неисправность стенда. Маловероятные дефекты.
%
%    \subpoint При наличии диагностического сообщения <<РК * - Ошибка цепи ds\_r\_h МКИ (проверка отсутствия замыкания)>> 
%	      (где <<*>> - номер разовой команды в диапазоне \{23..24\}) возможны следующие причины сбоя:
%	  \begin{enumerate}
%	    \item причины аналогичны перечисленным в ~\ref{subdefects:rk_out_0v_trig_fault_resl_cross}%Именно, ds_r_h соответствует resl!
%	    \item повреждение (дефект монтажа) разъема XP3 (VPX).
%	  \end{enumerate}
%	  
%    \subpoint При наличии диагностического сообщения <<РК * - Ошибка цепи ds\_r\_l МКИ (проверка отсутствия замыкания)>> 
%	      (где <<*>> - номер разовой команды в диапазоне \{23..24\}) возможны следующие причины сбоя:
%	  \begin{enumerate}
%	    \item причины аналогичны перечисленным в ~\ref{subdefects:rk_out_0v_trig_fault_resh_cross}%Именно, ds_r_l соответствует resh!
%	    \item повреждение (дефект монтажа) разъема XP3 (VPX).
%	  \end{enumerate}     
     
  \end{sloppypar}

%%%%- Группа Симптомов --------------------------------------------------------------------------------------------------
  \begin{sloppypar}
  \pointbold {Ошибка при выполнении теста <<РК 27В/Обрыв. Выходы.  Срабатывание>>}
    \subpoint При наличии диагностического сообщения <<РК * - Ошибка цепи ВСК resh>> 
	      (где <<*>> - номер разовой команды в диапазоне \{23..24\}) возможны следующие причины сбоя(ложного срабатывания датчика DS[*]RESH):
	\label{subdefects:rk_out_27v_trig_fault_resh}
	\begin{enumerate}
	  \item повреждение (дефект монтажа) компонентов каскада выходной разовой команды № <<*>>, образующих нижнее плечо, 
		то есть транзисторы и сервисные компоненты, подключающие цепь 0/27 В (см. \DocProductSignature~Э3);
	  \item повреждение (дефект монтажа) компонентов датчиков, формирующих сигналы DS[*]RESH или их сервисных компонентов, приводящее к ложному срабатыванию датчика DS[*]RESH;
	  \item повреждение (дефект монтажа) микросхемы ПЛИС DD30.
	\end{enumerate} 
	
    \subpoint При наличии диагностического сообщения <<РК * - Ошибка цепи ВСК resl>> 
	      (где <<*>> - номер разовой команды в диапазоне \{23..24\}) возможны следующие причины сбоя:
	\label{subdefects:rk_out_27v_trig_fault_resl}
	\begin{enumerate}
	  \item повреждение (дефект монтажа) компонентов каскада выходной разовой команды № <<*>>, образующих верхнее плечо, 
		то есть транзисторы и сервисные компоненты, подключающие цепь 27VF (см. \DocProductSignature~Э3);
	  \item повреждение (дефект монтажа) компонентов датчиков, формирующих сигналы DS[*]RESH или их сервисных компонентов, приводящее к ложному срабатыванию датчика DS[*]RESH;
	  \item повреждение (дефект монтажа) микросхемы ПЛИС DD30;
	  \item повреждение (дефект монтажа) предохранителя по цепи питания разовых команд модуля FU1.
	\end{enumerate}
    
     \subpoint При наличии диагностического сообщения <<РК * - Ошибка цепи ВСК resh (проверка отсутствия замыкания)>> 
	      (где <<*>> - номер разовой команды в диапазоне \{23..24\}) возможны следующие причины сбоя (ложного срабатывания датчика DS[*]RESH, относящегося к каналу, отличному от проверяемого):
	  \label{subdefects:rk_out_27v_trig_fault_resh_cross}
	  \begin{enumerate}
	    \item повреждение (дефект монтажа) компонентов каскадов разовых команд №23 и №24, 
		  приводящий к замыканию отдельных цепей, участвующих в формировании РК №<<*>>(см. \DocProductSignature~Э3);    
	    \item замыкание контактов микросхемы ПЛИС DD30.
	  \end{enumerate}
      
    \subpoint При наличии диагностического сообщения <<РК * - Ошибка цепи ВСК resl (проверка отсутствия замыкания)>> 
	      (где <<*>> - номер разовой команды в диапазоне \{23..24\}) возможны следующие причины сбоя (ложного срабатывания датчика DS[*]RESL, относящегося к каналу, отличному от проверяемого):
	  \label{subdefects:rk_out_27v_trig_fault_resl_cross}
	  \begin{enumerate}	     
	    \item повреждение (дефект монтажа) компонентов каскадов разовых команд №23 и №24, 
		  приводящий к замыканию отдельных цепей, участвующих в формировании РК №<<*>>(см. \DocProductSignature~Э3);
	    \item повреждение (дефект) печатной платы, приводящий к замыканию отдельных цепей, участвующих в формировании каналов №23 и №24;
	    \item замыкание контактов микросхемы ПЛИС DD30;
	    \item повреждение (дефект монтажа) разъема XP3 (VPX), приводящее к замыканию разовых команд №23 и №24.
	  \end{enumerate}
	  
    \subpoint При наличии диагностического сообщения <<РК * - Ошибка цепи ds\_r\_h МКИ>> 
	      (где <<*>> - номер разовой команды в диапазоне \{23..24\}) возможны следующие причины сбоя:    
	  \begin{enumerate}
	    \item причины аналогичны перечисленным в ~\ref{subdefects:rk_out_27v_trig_fault_resl};%Именно, ds_r_h соответствует resl!
	    \item повреждение (дефект монтажа) разъема XP3 (VPX).
	  \end{enumerate}
    
    \subpoint При наличии диагностического сообщения <<РК * - Ошибка цепи ds\_r\_l МКИ>> 
	      (где <<*>> - номер разовой команды в диапазоне \{23..24\}) возможны следующие причины сбоя:
	  \begin{enumerate}
	    \item причины аналогичны перечисленным в ~\ref{subdefects:rk_out_27v_trig_fault_resh};%Именно, ds_r_l соответствует resh!
	    \item повреждение (дефект монтажа) разъема XP3 (VPX).
	  \end{enumerate}

%%-- Эти пункты сложно объяснить. Ошибка может возникать при наличии смежной ошибки и означает неисправность стенда. Маловероятные дефекты.
%
%    \subpoint При наличии диагностического сообщения <<РК * - Ошибка цепи ds\_r\_h МКИ (проверка отсутствия замыкания)>> (где <<*>> - номер разовой команды в диапазоне \{23..24\}) возможны следующие причины сбоя:
%    \subpoint При наличии диагностического сообщения <<РК * - Ошибка цепи ds\_r\_l МКИ (проверка отсутствия замыкания)>> (где <<*>> - номер разовой команды в диапазоне \{23..24\}) возможны следующие причины сбоя:
  
  \end{sloppypar}
      
%%%%- Группа Симптомов --------------------------------------------------------------------------------------------------
  \pointbold {Ошибка при выполнении теста <<РК Готовность. 0В/Обрыв. Срабатывание>>}
    \subpoint При наличии диагностического сообщения <<РК 12 - МКИ не детектировала сигнал исправности>> возможны следующие причины сбоя:
	\begin{enumerate}
	  \item повреждение (дефект монтажа) компонентов каскада выходной разовой команды №12, 
		то есть транзистор и сервисные компоненты, подключающие цепь 0/27 B (см. \DocProductSignature~Э3);
	  \item повреждение (дефект монтажа) микросхемы ПЛИС DD30;
	  \item повреждение (дефект монтажа) стабилитрона VD22, C420, C421, формирующих напряжение 8 В;
	  \item повреждение (дефект монтажа) предохранителя по цепи питания разовых команд модуля FU1;
	  \item повреждение (дефект монтажа) разъема XP3 (VPX).
	\end{enumerate} 
	
    \subpoint При наличии диагностического сообщения <<РК 12 - МКИ не детектировала отсутствие сигнала исправности>> возможны следующие причины сбоя:
	\begin{enumerate}
	  \item повреждение (дефект монтажа) компонентов каскада выходной разовой команды №12: VT4, R382, C333, VD27;
	  \item повреждение (дефект монтажа) разъема XP3 (VPX).
	\end{enumerate}
      
%%%%- Группа Симптомов --------------------------------------------------------------------------------------------------
  \pointbold {Ошибка при выполнении теста <<РК Низковольтные. Срабатывание>>}
    \subpoint При наличии диагностического сообщения <<Ошибка - ВИМ-3U-3 не детектирует сигнал * >> 
	      (где <<*>> - номер разовой команды в диапазоне \{13..14\}) возможны следующие причины сбоя (отсутствия уровня логической <<1>>):
	\label{subdefects:rk_lv_fault_unexpected0}
	\begin{enumerate}
	  \item повреждение (дефект монтажа) компонентов каскада входной разовой команды DS\_LV[*], 
		а именно DD26 и сервисных компонентов (см. \DocProductSignature~Э3);
	  \item повреждение (дефект монтажа) микросхемы ПЛИС DD30;
	  \item повреждение (дефект монтажа) разъема XP3 (VPX).
	\end{enumerate}	
	
    \subpoint При наличии диагностического сообщения <<Ошибка - ВИМ-3U-3 детектирует ложное срабатывание сигнала *>> возможны следующие причины сбоя:
	      (где <<*>> - номер разовой команды в диапазоне \{13..14\}) возможны следующие причины сбоя (неправомерного наличия уровня логической <<1>>):
	\begin{enumerate}
	  \item причины аналогичны перечисленным в ~\ref{subdefects:rk_lv_fault_unexpected0}.
	\end{enumerate}
      
%%%%- Группа Симптомов --------------------------------------------------------------------------------------------------
  \pointbold {Ошибка при выполнении теста <<Интрерфейсные сигналы XMC. JN-C(12)..JN-C(19), JN-F(12)..JN-F(19). Проверка срабатывания>>}
    \subpoint При наличии диагностического сообщения <<Ошибка - МКИМ не детектирует JN*>>, 
	      предваряемого информационным сообщением <<Проверка мезонинных сигналов JN*>> 
	      (где <<*>> - номер проверяемой пары однопроводных сигналов JN-C(*)--JN-F(*)) возможны следующие причины сбоя:
      \label{subdefects:xmc_se12_se19_signals_fault_unexpected0}
      \begin{enumerate}
	\item повреждение (дефект монтажа) разъема XS2 (XMC);
	\item повреждение (дефект монтажа) разъема XP3 (VPX);
	\item плохой контакт разъёма (разъёмов) стендового мезонина с разъёмом (разъёмами) \DocProductShortTitle.
      \end{enumerate}
    
    \subpoint При наличии диагностического сообщения <<Ошибка - МКИМ детектирует ложное срабатывание бита **>>, 
	      предваряемого информационным сообщением <<Проверка мезонинных сигналов JN*>> 
	      (где <<*>> - номер первой проверяемой пары однопроводных сигналов JN-C(*)--JN-F(*), <<**>> - номер второй проверяемой пары однопроводных сигналов JN-C(**)--JN-F(**)) возможны следующие причины сбоя:
      \begin{enumerate}
	\item причины аналогичны перечисленным в ~\ref{subdefects:xmc_se12_se19_signals_fault_unexpected0}.
      \end{enumerate}
      
%%%%- Группа Симптомов --------------------------------------------------------------------------------------------------
  %\begin{sloppypar}
  \exhyphenpenalty=10000  %Запрет переноса слов с дефисом
  \pointbold {Ошибка при выполнении теста <<Сигналы XMC, служебные. Срабатывание>> (<<Интерфейсные сигналы XMC. JN-C(8)..JN-C(11), JN-F(8)..JN-F(11). Проверка срабатывания>>}
     \subpoint Во время данной проверки за счёт общих аппаратных ресурсов стендового мезонинного модуля одновременно проверяются (см. \DocStendMezSignature~Э3):
      \begin{enumerate}
	\item служебные сигналы мезонинного модуля:
	  \begin{enumerate}
	    \item выходные: MRSTI\_BUF\#, MVMRO\_BUF, MROOT\_BUF\#;
	    \item входные: MRSTO\_BUF\#, MPRESENT\_BUF\#, MBIST\_BUF\#, MWAKE\_BUF\#;
	  \end{enumerate}
	\item сквозные однопроводные сигналы мезонинного модуля: JN-C[8]..JN-C[11], JN-F[8]..JN-F[11].
      \end{enumerate}      
     
    \subpoint При применении стендового мезонина \DocStendMezSignature служебные выходные сигналы XMC следующим образом используются 
	      для проверки cквозных однопроводныхе мезонинных сигналов \DocProductShortTitle:
      \begin{enumerate}
	\item для проверки JN-C[8]  используется выход \DocProductShortTitle MRSTI\_BUF\#;
	\item для проверки JN-C[9]  используется инверсия от выхода \DocProductShortTitle MROOT\_BUF\#;
	\item для проверки JN-C[10] используется выход \DocProductShortTitle MVMRO\_BUF;
	\item для проверки JN-C[11] используется выход \DocProductShortTitle MROOT\_BUF\#.
      \end{enumerate}      
    
    \subpoint При применении стендового мезонина \DocStendMezSignature служебные входные сигналы XMC следующим образом используются 
	      для проверки cквозных однопроводныхе мезонинных сигналов \DocProductShortTitle:
      \begin{enumerate}
	\item для проверки JN-F[8]  используется вход \DocProductShortTitle MRSTO\_BUF\#;
	\item для проверки JN-F[9]  используется вход \DocProductShortTitle MPRESENT\_BUF\#;
	\item для проверки JN-F[10] используется вход \DocProductShortTitle MBIST\_BUF\#;
	\item для проверки JN-F[11] используется вход \DocProductShortTitle MWAKE\_BUF\#.
      \end{enumerate}    
    
    \subpoint При наличии диагностического сообщения <<Ошибка - ВИМ-3U-3 не детектирует JN *>>, 
	      предваряемого информационным сообщением <<Проверка мезонинных сигналов JN*>> и <<Проверка входов>>
	      (где <<*>> - номер первой проверяемой пары однопроводных сигналов JN-C(*)--JN-F(*)) возможны следующие причины сбоя:
      \label{subdefects:xmc_seF8_seF11_signals_fault_unexpected0}
      \begin{enumerate}
	\item повреждение (дефект монтажа) разъема XS2 (XMC);
	\item повреждение (дефект монтажа) разъема XP3 (VPX);
	\item повреждение (дефект монтажа) микросхемы ПЛИС DD30;
	\item плохой контакт разъёма (разъёмов) стендового мезонина с разъёмом (разъёмами) \DocProductShortTitle.
      \end{enumerate}
	      
    \subpoint При наличии диагностического сообщения <<Ошибка - ВИМ-3U-3 детектирует ложное срабатывание бита **>>,
	      предваряемого информационным сообщением <<Проверка мезонинных сигналов JN*>> и <<Проверка входов>>
	      (где <<*>> - номер первой пары <<JN-F(*)--входной сигнал XMC>>, а  <<**>> - номер второй пары <<JN-F(*)--входной сигнал XMC>> проверяемых однопроводных сигналов) возможны следующие причины сбоя:
      \begin{enumerate}
	\item причины аналогичны перечисленным в ~\ref{subdefects:xmc_seF8_seF11_signals_fault_unexpected0}.
      \end{enumerate}
      
    \subpoint При наличии диагностического сообщения <<Ошибка - МКИМ не детектирует JN *>>,
	      предваряемого информационным сообщением <<Проверка мезонинных сигналов JN*>> и <<Проверка выходов>>
	      (где <<*>> - номер первой проверяемой пары однопроводных сигналов JN-C(*)--JN-F(*)) возможны следующие причины сбоя:
      \begin{enumerate}
	\item причины аналогичны перечисленным в ~\ref{subdefects:xmc_seF8_seF11_signals_fault_unexpected0}.
      \end{enumerate}
      
    \subpoint При наличии диагностического сообщения <<Ошибка - МКИМ детектирует ложное срабатывание бита **>>, 
	      предваряемого информационным сообщением <<Проверка мезонинных сигналов JN*>> и <<Проверка выходов>>
	      (где <<*>> - номер первой пары <<JN-С(*)--выходной сигнал XMC>>, а  <<**>> - номер второй пары <<JN-С(*)--выходной сигнал XMC>> проверяемых однопроводных сигналов) возможны следующие причины сбоя:
      \begin{enumerate}
	\item причины аналогичны перечисленным в ~\ref{subdefects:xmc_seF8_seF11_signals_fault_unexpected0}.
      \end{enumerate}
      
    \subpoint При наличии диагностического сообщения <<Ошибка - МКИМ не детектирует JN9>>,
	      предваряемого информационным сообщением <<Проверка мезонинных сигналов JN11>> и <<Проверка выходов>> возможны следующие причины сбоя:
      \begin{enumerate}
	\item причины аналогичны перечисленным в ~\ref{subdefects:xmc_seF8_seF11_signals_fault_unexpected0}.
      \end{enumerate}
  
  %\end{sloppypar}
      
%%%%- Группа Симптомов --------------------------------------------------------------------------------------------------
  \begin{sloppypar}
  \pointbold {Ошибка при выполнении теста <<Сигналы VPX, однонаправленные. Срабатывание>>}
    \subpoint При наличии диагностического сообщения <<Ошибка - ВИМ-3U-3 не детектирует бит *>>, 
	      предваряемого информационным сообщением с именем сигнала (GA*, где <<*>> - номер бита географической адресации модуля от 0 до 4) возможны следующие причины сбоя:
      \label{subdefects:vpx_in_ga_fault_unexpected0}
      \begin{enumerate}
	\item повреждение (дефект монтажа) компонентов входного каскада сигналов GA[*]\#, а именно DD23 и сервисных компонентов (см. \DocProductSignature~Э3);
	\item повреждение (дефект монтажа) микросхемы ПЛИС DD30;
	\item повреждение (дефект монтажа) разъема XP3 (VPX).
      \end{enumerate}
      
    \subpoint При наличии диагностического сообщения <<Ошибка - ВИМ-3U-3 детектирует ложное срабатывание бита *>>, 
	      предваряемого информационным сообщением с именем сигнала (GA*, где <<*>> - номер бита географической адресации модуля от 0 до 4) возможны следующие причины сбоя:
      \begin{enumerate}
	\item причины аналогичны перечисленным в ~\ref{subdefects:vpx_in_ga_fault_unexpected0}.
      \end{enumerate}
    
    \subpoint При наличии диагностического сообщения <<Ошибка - ВИМ-3U-3 не детектирует сигнал>>,
	      предваряемого информационным сообщением с именем сигнала (GAP, SYSCON) возможны следующие причины сбоя:
      \label{subdefects:vpx_in_gap_syscon_fault_unexpected0}
      \begin{enumerate}
	\item повреждение (дефект монтажа) компонентов входного каскада сигналов GAP\#, SYS\_CON, а именно DD23 и сервисных компонентов (см. \DocProductSignature~Э3);
	\item повреждение (дефект монтажа) микросхемы ПЛИС DD30;
	\item повреждение (дефект монтажа) разъема XP3 (VPX).
      \end{enumerate}
      
    \subpoint При наличии диагностического сообщения <<Ошибка - ВИМ-3U-3 обнаружило ложное срабатывание сигнала>>,
	      предваряемого информационным сообщением с именем сигнала (GAP, SYSCON) возможны следующие причины сбоя:
      \begin{enumerate}
	\item причины аналогичны перечисленным в ~\ref{subdefects:vpx_in_gap_syscon_fault_unexpected0}.
      \end{enumerate}  
      
  \end{sloppypar}  
%%%%- Группа Симптомов --------------------------------------------------------------------------------------------------
  \begin{sloppypar}
  \pointbold {Ошибка при выполнении теста <<Сигналы VPX, двунаправленные, входы. Срабатывание>>}
    
    \begin{footnotesize}    
      Примечание --- Номер проверяемого входного сигнала VPX следующим образом соответствует имени цепи (см. \DocProductSignature~Э3):
      \begin{enumerate}
	\item 0: NVMRO;
	\item 1: MASKABLERESET\#;
	\item 2: SYSRESET\#;
	\item 3: GP\_OPMODE.
      \end{enumerate}      
    \end{footnotesize}
    
    \subpoint При наличии диагностического сообщения <<Ошибка - ВИМ-3U-3 не детектирует сигнал линии *>>
	      (где <<*>> - номер проверяемого входного сигнала) возможны следующие причины сбоя (отсутствия уровня логической <<1>>):
    \label{subdefects:vpx_bidir_in_fault_unexpected0}
      \begin{enumerate}
	\item повреждение (дефект монтажа) компонентов входного каскада сигналов NVMRO, MASKABLERESET\#, SYSRESET\#, GP\_OPMODE,
	      а именно DD23, DD26 и сервисных компонентов (см. \DocProductSignature~Э3);
	\item повреждение (дефект монтажа) микросхемы ПЛИС DD30;
	\item повреждение (дефект монтажа) разъема XP3 (VPX).
      \end{enumerate}
      
    \subpoint При наличии диагностического сообщения <<Ошибка - ВИМ-3U-3 обнаружил ложное срабатывание сигнала в линии *>>
	      (где <<*>> - номер проверяемого входного сигнала) возможны следующие причины сбоя (неправомерного наличия уровня логической <<1>>):
      \begin{enumerate}
	\item причины аналогичны перечисленным в ~\ref{subdefects:vpx_bidir_in_fault_unexpected0}.
      \end{enumerate}
    
  \end{sloppypar}
       
%%%%- Группа Симптомов --------------------------------------------------------------------------------------------------
  \begin{sloppypar}
  \pointbold {Ошибка при выполнении теста <<Сигналы VPX, двунаправленные, выходы. Срабатывание>>}
  
    \begin{footnotesize}    
      Примечание --- Номер проверяемого выходного сигнала VPX следующим образом соответствует имени цепи (см. \DocProductSignature~Э3):
      \begin{enumerate}
	\item 0: NVMRO;
	\item 1: MASKABLERESET\#;
	\item 2: SYSRESET\#;
	\item 3: GP\_OPMODE.
      \end{enumerate}      
    \end{footnotesize}
  
    \subpoint При наличии диагностического сообщения <<Ошибка - МКИ не детектирует сигнал линии *>>
	      (где <<*>> - номер проверяемого выходного сигнала) возможны следующие причины сбоя (отсутствия уровня логической <<1>>):
      \label{subdefects:vpx_bidir_out_fault_unexpected0}
      \begin{enumerate}
	\item повреждение (дефект монтажа) компонентов выходного каскада сигналов NVMRO, MASKABLERESET\#, SYSRESET\#, GP\_OPMODE,
	      а именно DD21, DD21 и сервисных компонентов (см. \DocProductSignature~Э3);
	\item повреждение (дефект монтажа) микросхемы ПЛИС DD30;
	\item повреждение (дефект монтажа) разъема XP3 (VPX).
      \end{enumerate}
      
    \subpoint При наличии диагностического сообщения <<Ошибка - МКИ детектирует ложное срабатывание сигнала линии *>>
	      (где <<*>> - номер проверяемого выходного сигнала) возможны следующие причины сбоя (неправомерного наличия уровня логической <<1>>):
      \begin{enumerate}
	\item причины аналогичны перечисленным в ~\ref{subdefects:vpx_bidir_out_fault_unexpected0}.
      \end{enumerate}
   
   \end{sloppypar}

%%%%- Группа Симптомов --------------------------------------------------------------------------------------------------
  \begin{sloppypar}
  \pointbold {Ошибка при выполнении теста <<Проверка метки времени>>}
    \subpoint При наличии диагностического сообщения <<Время ожидания ответа вышло>> возможны следующие причины сбоя:
      \begin{enumerate}
	\item запрограммирован несоответствующий загрузочный код ПЛИС;
	\item повреждение (дефект монтажа) микросхемы усилителя DD26 или её сервисных компонентов;
	\item повреждение (дефект монтажа) микросхемы ПЛИС DD30;
	\item повреждение (дефект монтажа) микросхемы ПЛИС DD1 (вывод B21);
	\item повреждение (дефект монтажа) разъема XP3 (VPX);
	\item повреждение (дефект монтажа) линии связи от XP3 (VPX) к DD26 (AUX\_CLK\_SE);
	\item повреждение (дефект монтажа) линии связи от DD26 к DD30 (TMARK\_SE);
	\item повреждение (дефект монтажа) линии связи от DD30 к DD1 (IRQ\_8);
	\item отсутствие отклика по технологическому интерфейсу RS232C.
      \end{enumerate}
  \end{sloppypar}      

%%%%-----------------------------------------------------------------------------------------------------------------------

%%---------------------------------------------------------------------------------------------------
\begin{comment}
%\pointbold{Возможные неисправности и методы их корректировки}
\section{Возможные неисправности и методы их корректировки}
\label{failure_reasons}  
UART[0]\_SOUT K49
UART[0]\_SIN K46  
UART[2]\_SOUT K53
UART[2]\_SIN K42
UART[3]\_SOUT K54
UART[3]\_SIN K43  
IIC[1]\_SDA K47
IIC[1]\_SCL K44  
IIC[2]\_SDA K55
IIC[2]\_SCL K50  
SPI\_MOSI K48
SPI\_MISO K45
SPI\_CLK K56
SPI\_CS0\# K51  
NVRAM1\_CS\#
NVRAM2\_CS\#  
IIC\_INT\_SDA  K57
IIC\_INT\_SCL K60  
SM0
SM1  
MSDA
MSCL  
\end{comment}
%%---------------------------------------------------------------------------------------------------