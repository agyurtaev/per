%% file: fpga_utility_usage.tex
  \section{Применение утилиты \emph{fpga} } %для обнаружения неисправностей на плате \DocProductShortTitle~}
  \label{sec:fpga_utility_usage}
  %\lstset{language=sh}
   
  \pointbold{Общие сведения об утилите \emph{fpga}}
    \subpoint При обнаружении неисправностей можно воспользоваться утилитой \emph{fpga}, входящей в состав стендового ПО модуля \DocProductShortTitle~.
    \subpoint С помощью утилиты \emph{fpga} можно считывать и записывать регистры и отдельные битовые поля ПЛИС, 
    используя их мнемоники (краткие обозначения) или адреса в десятичной форме (только для регистров).
    \subpoint Распределение адресного пространства ПЛИС и описание её функционирования приведено в документе \DocProductSignature~Д4 <<Перечень аппаратных ресурсов>>.
  
  \pointbold{Подготовка к запуску утилиты \emph{fpga}}
    
  \subpoint Убедитесь, что \DocStendShortTitle~ подготовлен к работе согласно пункту 3.2.4 руководства по эксплуатации \DocStendSignature~ РЭ.
  \subpoint Запустите терминальную программу:
    \begin{enumerate}
  \item Откройте терминальную программу putty.exe двойным щелчком мыши по ярлыку «putty» на рабочем столе стендового компьютера.
  \item Запустите терминальную сессию со следующими параметрами, нажав кнопку «Open» в окне программы «putty.exe»:
  \begin{enumerate}
    \item Тип соединения: Serial
    \item Имя порта: COM1
    \item Скорость: 115200
    \item Количество бит данных: 8
    \item Количество стоп-бит: 1
    \item Контроль четности: ОТКЛ.
    \item Управление потоком: ОТКЛ.
  \end{enumerate}
\end{enumerate}
  \subpoint Включите источники питания \DocStendShortTitle. Для этого нажмите на кнопку <<Output>> на приборах  <<А1 Источник питания APS-7205L>> и <<А4 Источник питания APS-7205L>> 
	и убедитесь, что надписи около <<Ch1>>, <<Ch2>> и <<Ch3>> изменились с <<OFF>> на <<ON>>.
  \subpoint Дождитесь окончания загрузки операционной системы. Признаком окончания загрузки является появление текста приглашения командной строки в строке окна терминальной программы: <</ \#>>.

	    
  \pointbold{Примеры использования утилиты \emph{fpga}}
    \subpoint Пример вызова утилиты \emph{fpga} для вывода карты доступных битовых полей (их мнемоники, адреса регистров, номера используемых битов):
      \begin{lstlisting}
	/tests/fpga.sh fmap
      \end{lstlisting}
    
    \subpoint Пример вызова утилиты \emph{fpga} для чтения битового поля <<Версия кода ПЛИС>>:
      \begin{lstlisting}
	/tests/fpga.sh read FW_VER
      \end{lstlisting}
      
    \subpoint Пример вызова утилиты \emph{fpga} для записи битового поля <<Разрешение формирования прерывания по приходу сигнала метки времени>>:
      \begin{lstlisting}
	/tests/fpga.sh write TMARK_CTRL.INT_EN 1
      \end{lstlisting}